\section{Future work}
There are a number of potential future extensions:
\begin{itemize}
  \item Running experiments on half of the entire population is expensive. If the treatment fraction could be reduced, the cost savings would often surpass the potential loss in statistical power. This can be achieved, for example, by constructing supergeo triplets, quadruplets, etc.~and assigning only one supergeo in each group to treatment. 
  %\item supergeo design to reduce variance of trimmed match estimator
  \item The current MIP formulation can be extended to include additional matching variables such as: socio-demographic characteristics, distance between geos, etc. Geo aggregation will need to be adjusted accordingly.
  \item The current design does not explicitly account for proximity of the regions, or travel patterns introducing additional interference/contamination. These considerations can be taken into account when constructing supergeos.
\end{itemize}

%We have proposed a new design methodology which generalizes the standard matched pairs design by allowing the flexibility to aggregate multiple experimental units into a super-unit. The new design is shown to significantly improve the matching quality for geographical experiments, which is nowadays the gold standard for online advertising. Our work naturally raises a few interesting open problems for future research:...