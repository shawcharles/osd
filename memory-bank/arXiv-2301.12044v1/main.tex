%%\documentclass{article}
\pdfoutput=1
\documentclass[12pt,fleqn]{article}
\setlength{\topmargin}{-0.25in}
\setlength{\textheight}{8.75in}
\setlength{\evensidemargin}{-0.125in}
\setlength{\oddsidemargin}{-0.125in}
\setlength{\textwidth}{6.75in}
\linespread{1.5}

% if you need to pass options to natbib, use, e.g.:
%     \PassOptionsToPackage{numbers, compress}{natbib}
% before loading neurips_2021

% ready for submission
%\usepackage{neurips_2021}

% to compile a preprint version, e.g., for submission to arXiv, add add the
% [preprint] option:
%     \usepackage[preprint]{neurips_2021}

% to compile a camera-ready version, add the [final] option, e.g.:
%     \usepackage[final]{neurips_2021}

% to avoid loading the natbib package, add option nonatbib:
%    \usepackage[nonatbib]{neurips_2021}

\usepackage[utf8]{inputenc} % allow utf-8 input
\usepackage[T1]{fontenc}    % use 8-bit T1 fonts
\usepackage{hyperref}       % hyperlinks
\usepackage{url}            % simple URL typesetting
\usepackage{booktabs}       % professional-quality tables
\usepackage{amsfonts}       % blackboard math symbols
\usepackage{nicefrac}       % compact symbols for 1/2, etc.
\usepackage{microtype}      % microtypography
\usepackage{xcolor}         % colors
\usepackage[labelfont=bf]{caption}

\usepackage{natbib}

\usepackage{amsmath}
\usepackage{amsthm}
\usepackage{amssymb}
\usepackage{graphicx}
%\usepackage{subcaption}
\usepackage{dsfont}
\usepackage{multirow}
\usepackage{subfigure}

\newtheorem{theorem}{Theorem}
\newtheorem{proposition}{Proposition}
\newtheorem{assumption}[theorem]{Assumption}

\def\var{\mrm{var}} % Variance 
\newcommand{\E}{{\mathbb{E}\,}}
\newcommand{\Var}{\mathrm{Var}}

\author{%
  \small Aiyou Chen \\
  \small Google \\
  \and
  \small Nick Doudchenko \\
  \small Google \\
  \and
  \small Shunhua Jiang \\
  \small Columbia \\
  \and
  \small Cliff Stein \\
  \small Google \\
  \and
  \small Bicheng Ying \\
  \small Google \\
}

\title{Supergeo Design: Generalized Matching for Geographic Experiments}

\makeatletter
\newcommand{\printfnsymbol}[1]{%
  \textsuperscript{\@fnsymbol{#1}}%
}
\makeatother


% The \author macro works with any number of authors. There are two commands
% used to separate the names and addresses of multiple authors: \And and \AND.
%
% Using \And between authors leaves it to LaTeX to determine where to break the
% lines. Using \AND forces a line break at that point. So, if LaTeX puts 3 of 4
% authors names on the first line, and the last on the second line, try using
% \AND instead of \And before the third author name.

\begin{document}
\maketitle

% Supergeo Design: Generalized Paired Matching Design
% keywords: causal inference, experimental design, online advertising

% It is OKAY to include author information, even for blind
% submissions: the style file will automatically remove it for you
% unless you've provided the [accepted] option to the icml2023
% package.

% List of affiliations: The first argument should be a (short)
% identifier you will use later to specify author affiliations
% Academic affiliations should list Department, University, City, Region, Country
% Industry affiliations should list Company, City, Region, Country

% You can specify symbols, otherwise they are numbered in order.
% Ideally, you should not use this facility. Affiliations will be numbered
% in order of appearance and this is the preferred way.
% \icmlsetsymbol{equal}{*}

% \begin{icmlauthorlist}
% % \icmlauthor{Firstname1 Lastname1}{equal,yyy}
% % \icmlauthor{Firstname2 Lastname2}{equal,yyy,comp}
% % \icmlauthor{Firstname3 Lastname3}{comp}
% % \icmlauthor{Firstname4 Lastname4}{sch}
% % \icmlauthor{Firstname5 Lastname5}{yyy}
% % \icmlauthor{Firstname6 Lastname6}{sch,yyy,comp}
% % \icmlauthor{Firstname7 Lastname7}{comp}
% % %\icmlauthor{}{sch}
% % \icmlauthor{Firstname8 Lastname8}{sch}
% % \icmlauthor{Firstname8 Lastname8}{yyy,comp}
% %\icmlauthor{}{sch}
% %\icmlauthor{}{sch}
% \icmlauthor{Aiyou Chen}{aiyou}
% \icmlauthor{Nick Doudchenko}{nick}
% \icmlauthor{Shunhua Jiang}{shunhua}
% \icmlauthor{Cliff Stein}{cliff}
% \icmlauthor{Bicheng Ying}{bicheng} 
% \end{icmlauthorlist}



% \icmlaffiliation{aiyou}{Google (now at Waymo), Mountain View, CA 94043.} %\texttt{aiyouchen@google.com}}
% \icmlaffiliation{nick}{Google, Boulder, CO 80301.}
% %\texttt{nikolayd@google.com}}
% \icmlaffiliation{shunhua}{Columbia University, New York, NY 10027. Work done while doing an internship at Google.}
% %\texttt{sj3005@columbia.edu}}
% \icmlaffiliation{cliff}{Google, New York, NY 10011.}
% %\texttt{cstein@google.com}}
% \icmlaffiliation{bicheng}{Google, Los Angeles, CA 90291.}
% %\texttt{ybc@google.com}}
% \icmlcorrespondingauthor{Nick Doudchenko}{nikolayd@google.com}

% % You may provide any keywords that you
% % find helpful for describing your paper; these are used to populate
% % the "keywords" metadata in the PDF but will not be shown in the document
% \icmlkeywords{causal inference, experimental design, online advertising}

% \vskip 0.3in
% ]

% % this must go after the closing bracket ] following \twocolumn[ ...

% % This command actually creates the footnote in the first column
% % listing the affiliations and the copyright notice.
% % The command takes one argument, which is text to display at the start of the footnote.
% % The \icmlEqualContribution command is standard text for equal contribution.
% % Remove it (just {}) if you do not need this facility.

% %\printAffiliationsAndNotice{}  % leave blank if no need to mention equal contribution
% \printAffiliationsAndNotice{\icmlEqualContribution} % otherwise use the standard text.

\begin{abstract}
\noindent We propose a generalization of the standard matched pairs design %\Shunhua{Change the name to paired matching design?}\Aiyou{+1 as no other paper mention 'mimum-weight matching experimental design'} 
in which experimental units (often geographic regions or \emph{geos}) may be combined into larger units/regions called ``supergeos'' in order to improve the average matching quality. % without excluding geos which would lead to poorly matched pairs. 
Unlike optimal matched pairs design which can be found in polynomial time \citep{lu2011optimal}, this generalized matching problem is NP-hard. We formulate it as a mixed-integer program (MIP) and show that experimental design obtained by solving this MIP can often provide a significant improvement over the standard design regardless of whether the treatment effects are homogeneous or heterogeneous. Furthermore, we present the conditions under which trimming techniques that often improve performance in the case of homogeneous effects \citep{chen2022robust}, may lead to biased estimates and show that the proposed design does not introduce such bias. We use empirical studies based on real-world advertising data to illustrate these findings.
%When the treatment effects are homogeneous, the performance of the proposed method is comparable (usually better) \Shunhua{Usually better is to be too strong. I would remove it.} than that of the \emph{Trimmed Match Design} (TMD) \Shunhua{Trimmed match estimator?} from \citet{chen2021trimmed} which excludes (i.e.~\emph{trims}) some of the poorly matched pairs in order to reduce the variance of the estimate. When the effects are heterogeneous, the proposed method may substantially outperform TMD for estimating the average treatment effect across the entire population.\Aiyou{A comparison with classical matched pairs design would have more impact than with TMD, as revised above; Please feel free to revert.}
% We also introduce a number of heuristics that allow solving the problem with hundreds of units without much detriment to the statistical performance. 
%\Shunhua{The abstract seems a bit long to me. Can we make it more concise?}\Aiyou{Shortened a little.}\Shunhua{LGTM}
% Numerical studies based on real data are reported.
\end{abstract}

% \setlength{\abovedisplayskip}{3pt}
% \setlength{\belowdisplayskip}{3pt}

\section{Introduction}\label{intro}
With online advertising revenue in the US amounting to almost 200 billion dollars in 2021 \citep{iab_report_2021}, it is both theoretically and practically important to understand advertising effectiveness. For instance, researchers may want to know how much additional sales revenue does an additional dollar spend on advertising generate? %\Nick{Following the example of one of Aiyou's earlier papers, found a reference that has a number for the size of ``online'' advertising industry. I'm curious if we can either find a similar number for ``all'' advertising or if it's okay to leave it as is, but continue talking in a bit more general way about advertising.} 
Ideally, such questions would be addressed via
the ``gold standard'' of causal inference---randomized experiments or A/B tests as they are often called in applied settings.  
%\Cliff{I replaced this sentence by the one above: To establish a causal relationship between advertising spend and sales revenue (or an alternative outcome variable), a researcher would be remiss not to contemplate utilizing the ``gold standard'' of causal inference---randomized experiments or A/B tests as they are often called in applied industrial settings.}

When many distinct units are available to experiment on and the estimand of interest is the \emph{average treatment effect} (ATE), simple randomized experiments may provide precise and intuitive results. Unfortunately, such experiments are not always available or, even when they are, may not be adequate for the question at hand. For instance, if interference between the experimental units---a situation in which the treatment status of one unit affects the outcomes of some other units---is not negligible, the researchers may decide to combine units into larger groups (or clusters) if they have reasons to believe that interference at the group level is less of a concern. In other cases, institutional constraints or privacy concerns may prevent the researchers from assigning the treatment at a more granular level. Neither of these concerns are foreign to the studies of advertising effectiveness \citep{coey2016people,vaver2011measuring}.   %\Cliff{Needs a citation}\Aiyou{Added two references} 
For example, product information learned from an ad by one person can be easily propagated to another person who has not seen the ad introducing interference. Different variations of television advertising are often assigned at the designated market area (DMA) level limiting the number of experimental units to only 210 DMAs existing in the US.\footnote{\url{https://markets.nielsen.com/us/en/contact-us/intl-campaigns/dma-maps/}} Moreover, targeted advertising is an important subject of current academic and public discourse with an increasing number of studies conducted at geographic---instead of more granular---levels \citep[Google Ads geo targets,\footnote{\url{https://developers.google.com/google-ads/api/reference/data/geotargets}}][]{rolnick2019randomized}. 
%\Nick{Not sure if we want to get into this, but might be a good idea to illustrate that the potential concerns mentioned earlier are relevant to iROAS estimation. Perhaps we could come up with a good non-controversial reference here.}\Shunhua{I vote for not going into more details here. This paragraph already looks convincing to me.}\Nick{The current sentence (after ``Moreover'') reads weird to me without any references. If we don't want to cite anything, I would just remove this sentence.}\Bicheng{Why not we just refer the GraphCut on Geo Adversiting paper and its reference for another example?}\Aiyou{It looks a good idea to add some references, e.g. Google Ads website on geo targeting? https://developers.google.com/google-ads/api/reference/data/geotargets} 
These considerations may not only limit the sample size, but also lead to experimental units that vary substantially along observed---and potentially unobserved---characteristics, thereby preventing the researchers from relying on large-sample properties alone and forcing them to make additional assumptions.

We are primarily motivated by evaluating advertising effectiveness using geographic experiments. However, these issues---relatively small sample size and heterogeneity of experimental units---may come up in other settings. Regardless of the setting, there are two general ways to improve the statistical properties of experimental results---by changing the allocation of units to treatment and control (\emph{experimental design}) or by changing the approach to estimating the quantity of interest (\emph{post-experimental analysis}). One common approach to experimental design is to find disjoint pairs of comparable units and randomize the treatment within each pair---the so called \emph{matched pairs design} \citep{imbens2015causal}. This experimental design has several advantages:  %\Shunhua{Change this sentence to ``This \emph{paired matching design} has several advantages''? In later sections I'm referring this approach as the paired matching design, and I think here is a good place to introduce this name.}\Nick{Done.}
(i) it allows---at least to some degree---balancing the control and treatment groups with respect to the observed unit-level covariates, (ii) the treatment effects can be estimated separately for each pair which may be particularly useful when the effects are heterogeneous, (iii) randomizing the treatment within each pair allows rigorous statistical inference (e.g.~permutation-based tests). Optimal matched pairs can be designed by solving a minimum-weight matching problem  %\Shunhua{Change this to minimum weight matching problem? Since that's the more common name in CS.} 
\citep[see, for example, Chapter~12 of][]{rosenbaum2020design} which can be done in polynomial time \citep{edmonds1965maximum}.\footnote{See \citet{stuart2010matching}, \citet{rosenbaum2020design} and \citet{pashley2021insights} for additional references to various general-purpose matching methods.}
%\Cliff{Should say here something like.  If the weight is defined by some similarity metric between two units -- for instance, ias the distance in the covariate space-- then the problem of finding the best pairing is a (non-bipartite) minimum weight matching problem.  This problem can be solved in polynomial time, e.g.  \citep{edmonds1965maximum} ... say more about statistics literature} In practice, the classical minimum-weight matching (MWM) algorithm, where the weight is defined by some similarity metric between two units---for instance, as the distance in the covariate space---can generate the design efficiently \citep{edmonds1965maximum}. \Nick{We need more references to applied statistical research.} 
%\Bicheng{I found this interesting literature: \href{https://journals.sagepub.com/doi/pdf/10.3102/1076998620946272}{Insights on Variance Estimation for Blocked and Matched Pairs Designs}: 
%\begin{quote} much of the prior work on randomized experiments has focused on two forms of blocking: blocking where there are several treated and control units in each block and blocking where there is exactly one treated and one control unit in each block (matched pairs). This literature, for the most part, has a gap: It has not extensively treated the cases where researchers have generated groups of varying size but where there is still only one treated and/or one control in some of the blocks, which we call the “hybrid design.” \end{quote} there are lots of literatures cited there as well.}\Nick{Added a reference.}

Unfortunately, it is not always possible to construct pairs of units that are sufficiently similar. It is not uncommon in applications to have pairs of units that are so different from each other that it is better to exclude some of them from the experiment completely in order to make the estimates more precise. This can be done either at the design phase \citep{chen2021trimmed} or during post analysis \citep{chen2022robust}. However, such ``trimming'' reduces the sample size and, perhaps more importantly, makes the resulting sample potentially different from the original population in terms of the average treatment effect if the effects are heterogeneous across units. Moreover, excluding pairs of units from the experiment may require stopping advertising in those units for the duration of the experiment. This may lead to a drop in sales revenue which otherwise could have been avoided.\footnote{We discuss the difference between ``trimming'' in the design and post-experimental analysis phases in a bit more detail in Section~\ref{sec:eval}.}%\Shunhua{I think we need to make it clear whether we are referring to trimmed match as an experimental design or a post-analysis tool here. We could add a footnote to make it clear that in the original paper trimmed match has both parts, but we consider them separately.}\Nick{I think it's fine in the current form for the introduction, but we should make it clear further on. Having said that, I'm open to suggestions on how to reword this.}\Cliff{Need to make clear that we want to keep all the data}\Aiyou{Good point. Tried to clarify it as above}\Nick{Added a footnote.}

\paragraph{Supergeo design.} To combat the issue of poor matches without sacrificing the data, we introduce a generalized matching problem in which each pair is a pair of ``superunits'' with each superunit being a sum of several original units. Conceptually, this extends the classical matched pairs to matched superunit-pairs.\footnote{The notion of ``superunit'' may not apply to all settings as the covariates and/or response variables may not be additive. For example, in medical science, it may not be meaningful to pair a group of two patients to another group of three patients. However, other aggregation approaches---besides the simple summation---may be considered depending on the application.} Since our primary motivation for this paper is geographic experiments where each unit is a DMA, in the remainder of the paper we use \emph{geo} and \emph{supergeo} as synonyms for unit and superunit respectively. We use the term \emph{supergeo design} to refer to the proposed methodology.

Finding the optimal supergeo design is closely related to the minimum-weight matching problem over the hyper-graph \citep{keevash2014geometric}. What differentiates our paper from previous algorithmic work is the fact that achieving good matches is not the end goal, but rather an approach to improving the statistical properties of the experimental results. Subsequently, we evaluate the proposed methodology as an experimental design tool.

We show that supergeo design can improve the precision of the average treatment effect estimates when compared to standard matched pairs design. %\Aiyou{Do we compare with simple randomized designs? If so, we need to add it back} %\Nick{Ideally, here we would add "and Trimmed Match Design" or whatever is the methodology that we consider for trimming some pairs before the experiment. I'll work on that.}
Moreover, it can substantially outperform the alternatives that require excluding some of the geos from the experiment when the treatment effects are heterogeneous since those alternative designs may lead to a subpopulation with a different average treatment effect compared to the original sample. 

Although further generalizations \citep[for example, in the spirit of the synthetic-control literature,][]{abadie2010synthetic} are possible, we currently assume that when the units are aggregated, the covariates used for matching are added up. For instance, this is appropriate when the covariate of interest is the total sales within a geo and we are willing to assume that as long as the time series of past sales data are matched well, the units should respond to treatment similarly.%\Aiyou{Is the point to say that if time series of past sales are matched well, their future sales would be comparable too if no treatment intervention would be applied?}\Nick{Yes, but also that if treated each of them would have a similar outcome under treatment.}

\paragraph{Contributions.} The main contributions of the paper are as follows:
\begin{itemize}
\item We propose a novel experimental design approach---supergeo design---that improves the balancing properties of matched pairs design. %\Shunhua{Should we use "matched pairs design" or "paired matching design"? Let's be consistent.}\Aiyou{Google search of 'paired matching design' leads to 'matched pairs design':) https://screenshot.googleplex.com/m7QWNJoGfv3HVk5. So revised.}\Shunhua{Thanks.}
\item We show that supergeo design can be particularly effective when the underlying treatment effects are heterogeneous since it allows achieving good treatment-control balance without excluding geos which would normally lead to poorly matched pairs.
%some of the poorly matched pairs from the experiment and/or analysis.
\item While this generalized matching problem is NP-hard, we propose an approach utilizing mixed-integer programming (MIP) that  solves the problem in practically important settings with hundreds of experimental units (representing geographic regions).
\item We validate the effectiveness of the proposed design using real advertising data.
\end{itemize}

%\Shunhua{We could also consider adding some bullet points here to summarize our contributions, as what is usually done in ICML/Neurips papers.} \Bicheng{+1, it is very common in these conference.}\Cliff{+1}

\paragraph{Structure of the paper.} We start by introducing a theoretical framework that allows us to discuss the statistical properties of alternative experimental design and analysis procedures. We then describe the methodology used for solving the supergeo matching problem. The empirical results section of the paper compares supergeo design to the matched pairs design. We also compare supergeo design to \emph{trimmed match estimator} \citep{chen2022robust} which is an analysis---rather than design---procedure and discuss the way in which supergeo design and trimmed match can be combined.%\footnote{In Section~\ref{sec:eval} we go into more detail on why this comparison is performed.}
\section{Model}\label{sec:model}
Let $\mathcal{G}$ denote the set of experimental units (geos) and let $N := |\mathcal{G}|$.\footnote{If $\mathcal{G}$ is the set of DMAs in the US, $N=210$.} For each $g \in \mathcal{G}$, $S_g$ is a variable controlled by the researcher and $R_g$ is the response variable presumably affected by $S_g$. For advertising experiments, $S_g$ and $R_g$ correspond to the advertising spend and returns respectively. Following the Neyman–Rubin causal framework \citep{imbens2015causal}, we use $(R_g^{(t)}, S_g^{(t)})$ and $(R_g^{(c)}, S_g^{(c)})$ to denote the potential outcomes under the treatment condition and the control condition, respectively. 

%\paragraph{Quantity of interest.}
The following ratio measures the unit-level causal effect of the treatment:
\begin{align*}
    \theta_g := \frac{R_g^{(t)} - R_g^{(c)}}{S_g^{(t)} - S_g^{(c)}},  \;\;\;\forall g \in \mathcal{G}.
\end{align*}
We are mainly interested in estimating the following ratio:
\begin{align*}
    \theta := \frac{\sum_g R_g^{(t)} - \sum_g R_g^{(c)}}{\sum_g S_g^{(t)} - \sum_g S_g^{(c)}},
\end{align*}
which measures the global causal response rate with respect to the controlled variable. In advertising experiments the quantity $\theta$ is often called the \emph{incremental return on ad spend} (iROAS)---the ratio of the additional return to the additional advertising spend \citep{chen2022robust}. If the response is measured in the same units (e.g.~USD) as the ad spend, $\theta>1$ indicates that it makes sense to expend an additional dollar on advertising.

\paragraph{Design and estimators.}
In practice we never observe both $(R_g^{(t)}, S_g^{(t)})$ and $(R_g^{(c)}, S_g^{(c)})$ since we can only assign a unit to either treatment or control. Let us use $\mathcal{T}$ and $\mathcal{C} \subseteq \mathcal{G}$ to denote the treatment and control groups in the experiment.
The standard way to estimate $\theta$ is to use an \emph{empirical estimator} that computes the ratio of differences between the treatment and control groups:
\begin{align}\label{eq:theta_hat}
\hat{\theta} := \frac{\sum_{g \in \mathcal{T}} R_g - \sum_{g \in \mathcal{C}} R_{g}}{\sum_{g \in \mathcal{T}} S_g - \sum_{g \in \mathcal{C}} S_{g}}.
\end{align}
The researcher might want to carefully choose the treatment group, $\mathcal{T}$, and the control group, $\mathcal{C}$, so that $\hat{\theta}$ is a good---in the statistical sense---estimate of $\theta$.\footnote{In the remainder of the paper we often refer to this experimental design stage as the \emph{pretest} phase as opposed to the \textit{test} phase which refers to the stage when the treatment is applied and the experimental data are collected and analyzed.}
%\Cliff{We should say here something about how our goal is to choose an $\mathcal{S}$ and $\mathcal{T}$ so that $\hat{\theta}$ is a good estimate of $\theta$.}\Shunhua{Done.}
% The experiment needs to be designed before the test starts and thus can only use data available in . We also refer to the actual experiment as the \textit{test} phase.
%The experiment consists of two phases. We call the first phase the \textit{pretest} phase, in which no experimental intervention is presented. The researchers collect the observed values in the pretest phase and use them to design the experiment. The second phase is called the \textit{test} phase, in which the researchers apply treatment to the treatment group according to the constructed design. 
For example, in the classical matched pairs design, geos are allocated into pairs, $\{(g_{k,+}, g_{k,-})\}_{k=1}^{N/2}$. Then, one of the geos in each pair is randomly assigned to the treatment group while the other one is assigned to the control group.
%Our proposed supergeo design is a natural generalization of the matched pairs design.



An alternative estimator that is often more precise than the empirical estimator is the \emph{trimmed match estimator} from \citet{chen2022robust}. This estimation procedure amounts to selecting the subsets $\mathcal{T}' \subseteq \mathcal{T}$ and $\mathcal{C}' \subseteq \mathcal{C}$ in order to obtain the most precise estimate of the form:\footnote{See \citet{chen2022robust} for the specifics of how the subsets $\mathcal{T}'$ and $\mathcal{C}'$ are selected.}

%\footnote{We remark that trimming can also be done in the experimental design phase as described in \cite{chen2021trimmed}. But with the knowledge of the data generated in the experiments, trimming in the analysis phase is at least as good as trimming in the design phase, so in this paper we only compare our supergeo design with the trimmed match estimator that performs trimming in analysis.} they trim the pairs that are not well-matched according to the experiment data, and for the remaining treatment geos $\mathcal{T}' \subseteq \mathcal{T}$ and control geos $\mathcal{C}' \subseteq \mathcal{C}$, compute
\begin{equation}\label{eq:theta_hat_trim}
\hat{\theta}^{\mathrm{trim}} := \frac{\sum_{g \in \mathcal{T}'} R_g - \sum_{g \in \mathcal{C}'} R_{g}}{\sum_{g \in \mathcal{T}'} S_g - \sum_{g \in \mathcal{C}'} S_{g}}.
\end{equation}
%The trimmed match estimator can be easily extended to our supergeo design, and in our evaluations we will compare the supergeo design with the matched pairs design under both the empirical estimator and the trimmed match estimator.
%\Shunhua{I added a description of matched pairs design and trimmed match estimator here. One question: Would readers be confused about why we only optimize the variance of $\hat{\theta}$ instead of $\hat{\theta}^{\mathrm{trim}}$? We could add something like our goal is to show supergeo design + empirical estimator is as good as trimmed match estimator.}\Cliff{I think the way you wrote this is good}

\paragraph{Model assumptions.}
We follow \cite{chen2022robust} and make a number of assumptions that facilitate a formal analysis of alternative experimental design and analysis procedures.

%Since we are interested in the ratio between responses and spends, which is inherently a linear quantity, the first and the most important assumption states that the observed response $R_g$ is linearly increasing with respect to the controlled variable $S_g$. 
\begin{assumption}[Linear response model]\label{ass:linear}
For each $g \in \mathcal{G}$ the \emph{uninfluenced response},\footnote{Corresponding to the response under zero advertising spend.} $Z_g$, does not depend on treatment assignment, and $R_g$ depends on $S_g$ linearly:
\begin{align*}
R_g = Z_g + \theta_g \cdot S_g.
\end{align*}
\end{assumption}

\noindent The next assumption stipulates that the total difference in spend between the treatment and control units is unaffected by the treatment assignment. It captures the real-life scenario in which the researcher has a budgetary constraint when conducting the experiment. It also simplifies the calculation of the variance of $\hat{\theta}$.
\begin{assumption}[Fixed budget]\label{ass:fixed_budget}
There is a fixed experimental budget $B$ so that
\[
\sum_{g \in \mathcal{T}} S_g - \sum_{g \in \mathcal{C}} S_{g} = B.
\]
\end{assumption}

% Finally, for simplicity we will first consider the case where all unit-level $\theta_g$'s are the same, and later extend to heterogeneous $\theta_g$'s.
% \begin{assumption}[Homogeneous $\theta_g$]\label{ass:homogeneous}
% For all $g \in \mathcal{G}$, $\theta_g = \theta$.
% \end{assumption}

%\Shunhua{TODO: add a more detailed description of trimmed match here.}




\paragraph{Supergeo design.}
%We propose the supergeo experimental design, which is a generalization of the standard matched pairs design. \Aiyou{Duplicated with the intro 'Supergeo matching design'; Shall we remove?}
%\Shunhua{The purpose of this paragraph is to introduce the formal notations $\{(G_{k,+}, G_{k,-})\}_{k=1}^K$ and $A_k$. I could try to shorten it.}

The supergeo experimental design proposed in this paper consists of $K$ \emph{supergeo pairs} $\{(G_{k,+}, G_{k,-})\}_{k=1}^K$ where $G_{k,+}, G_{k,-} \subset \mathcal{G}$ are nonempty sets of geos which we call supergeos. Note that if we restrict all supergeos to have size one, we recover the standard matched pairs design. %The optimal supergeo design is the design that minimizes the bias and the variance of the empirical estimate $\hat{\theta}$. In the next two sections we will analyze the bias and the variance of $\hat{\theta}$. %and in Section~\ref{sec:supergeo_algo} we propose an algorithm that uses mixed integer programming (MIP) to compute the optimal supergeo design using the pretest data.


In each supergeo pair we randomly assign one of the supergeos to the treatment group and the other one to the control group. %implying that all geos that belong to the treated supergeo are treated and, similarly, all geos that belong to the control supergeo are not treated. \Shunhua{The part after ``implying'' seems redundant to me.} 
Let $A_k \in \{\pm 1\}$ for $k=1,\dots,K$ be $i.i.d.$ random sign variables independent from the rest of the data. Let $A_k=1$ with probability $1/2$ indicating that $G_{k,+}$ is treated and $A_k=-1$ with probability $1/2$ indicating that $G_{k,-}$ is treated. 
%Then, the researchers conduct the experiment and apply the treatment to the treatment group, i.e., they design the control variables $S_g$ to be different for the treatment and the control group. In this paper we consider \textit{heavy up} experiments that increase $S_g$ for geos in the treatment group, and keep $S_g$ the same for geos in the control group.\footnote{The methods of this paper can be easily extended to another commonly used \emph{go dark} experiment which sets treatment $S_g$'s to zeroes.} Lastly, the researchers collect the observed response variables in the test phase and generate the reports including the empirical estimate $\hat{\theta}$ and confidence intervals.

For convenience, for a supergeo $G \subset \mathcal{G}$, we define the spend and response of $G$ as $S_{G} := \sum_{g \in G} S_g$ and $R_{G} := \sum_{g \in G} R_g$. Similarly, we define the uninfluenced response of $G$ as $Z_{G} := \sum_{g \in G} Z_g$. 


\subsection{Homogeneous $\theta_g$}\label{sec:homo_model}
We start by computing the bias and variance of the empirical estimator, $\hat{\theta}$, under the assumption that all geos respond to treatment in the same way.
\begin{assumption}[Homogeneous $\theta_g$]\label{ass:homogeneous}
For all $g \in \mathcal{G}$, $\theta_g = \theta$.
\end{assumption}

\noindent Note that the response difference between treatment and control of the $k$-th supergeo pair $(G_{k_+}, G_{k_-})$ can be expressed as $A_k (R_{G_{k,+}} - R_{G_{k,-}})$.
% we can establish
% \begin{align*}
%     \hat{\theta} =&\,\frac{\sum_{g \in \mathcal{T}} R_g - \sum_{g \in \mathcal{C}} R_{g}}{\sum_{g \in \mathcal{T}} S_g - \sum_{g \in \mathcal{C}} S_{g}}\\ %\frac{\sum_{k} R_{k,t} - R_{k,c}}{\sum_{k} C_{k,t} - C_{k,c}}\\
%     =&\, \frac{\sum_{k} A_k (R_{G_{k,+}} - R_{G_{k,-}})}{\sum_{k} A_k (S_{G_{k,+}} - S_{G_{k,-}})} \\
%     \stackrel{(a)}{=}&\, \theta +  \frac{\sum_{k} A_k (Z_{G_{k,+}} - Z_{G_{k,-}})}{\sum_{k} A_k (S_{G_{k,+}} - S_{G_{k,-}})} \\
%     \stackrel{(b)}{=}&\, \theta +  \frac{1}{B} \sum_{k} A_k (Z_{G_{k,+}} - Z_{G_{k,-}}),
% \end{align*}
% where step (a) follows from the linear model and the homogeneous $\theta$ assumption, and step (b) follows from the fixed budget assumption. 
Assumptions~\ref{ass:linear} and~\ref{ass:fixed_budget} imply that %\citep[see][for the derivation]{chen2021trimmed} \Shunhua{I would remove this reference since the TM paper doesn't explicitly have this formula. I think it's ok to directly present this formula here.}\Nick{Sounds good.}
\begin{align*}
     \hat{\theta} & =  \theta +  \frac{1}{B} \sum_{k} A_k (Z_{G_{k,+}} - Z_{G_{k,-}}).
\end{align*}
Since $\{A_k: k=1,\dots,K\}$ are $i.i.d.$~zero-mean random variables that are independent from the rest of the variables, we conclude that $\hat{\theta}$ is an \emph{unbiased} estimator of $\theta$: $\E[\hat{\theta}] = \theta$. The above equation also implies that the variance of $\hat{\theta}$ (note that the randomness is over the $A_k$'s) is:
%\Cliff{At this point the choice of the supergeos is irrelevant to the computation of $\hat{\theta}$, it only really matters whether a geo is in control or treatment.  We should point out that we are aware of this, or someone may stop reading because it looks trivial.}\Shunhua{The treatment and control groups are determined by the supergeo pairs + random assignment, so $\hat{\theta}$ would depend on the supergeos. Should I emphasize that the expectation and variance are over the randomness of the $A_k$'s?}\Cliff{Yes.  My point was that in the end, once you made the random choices, it doesn't matter what the supergeo's are.  But since we say we choose the supergeos first, I think just pointing out that the randomness is over the A's is good.}\Shunhua{Done.}
\begin{align}
    %\E[\hat{\theta}] =\,& \theta,  \notag \\
    \Var[\hat{\theta}] =\,& \frac{1}{B^2}\sum_{k}(Z_{G_{k,+}} - Z_{G_{k,-}})^2. \label{eq:variance}
\end{align}
The goal of our experimental design procedure is to find supergeo pairs that minimize $\sum_{k}(Z_{G_{k,+}} - Z_{G_{k,-}})^2$. Note that without a constraint on the maximum size of a supergeo pair, the optimal way to minimize the variance is to split all geos into a single supergeo pair $({G_{+}}, {G_{-}})$ with minimum difference $|Z_{G_{+}} - Z_{G_{-}}|$. We will add a size constraint that is important for several reasons that will be discussed in Section~\ref{sec:supergeo_algo}.
%\Cliff{Similar to last comment.  At this point we see why the supergeos matter, but the minimum variance comes from having just 2 supergeos.  We should point out that we are aware of this and will be adding other constraints.}\Shunhua{Done.}



\subsection{Heterogeneous $\theta_g$}\label{sec:hetero_model}
%\section{Extension to Heterogeneous Model}
In practice, it is quite possible that different geos respond to treatment differently. This leads us to a model in which Assumption~\ref{ass:homogeneous} is abandoned. 
%Assume that the linear model of Assumption~\ref{ass:linear} also holds for the potential outcomes, then
Note that the linearity Assumption~\ref{ass:linear} implies that
%Define $\Delta S_g := S_g^{(t)} - S_g^{(c)}$, then we have
\begin{align*}
    \theta = &~ \frac{\sum_g R_g^{(t)} - \sum_g R_g^{(c)}}{\sum_g S_g^{(t)} - \sum_g S_g^{(c)}}
    = \frac{\sum_g \theta_g \cdot (S_g^{(t)} - S_g^{(c)})}{\sum_g (S_g^{(t)} - S_g^{(c)})}.
\end{align*}
Therefore, the ratio of interest, $\theta$, depends on the exact way in which we change the controlled variable $S_g$ under the treatment and control conditions. 
A natural choice is to increase all treatment spends proportionally to the original (pre-experimental) spends and leave the control spends unaffected.%We make the following stronger assumption.
\begin{assumption}[Proportional spend]\label{ass:ideal_model}%\Aiyou{What about 'Proportional spend' instead of 'Ideal model'}
Suppose that each geo $g$ has the initial spend $\overline{S}_g$. In the experiment, if $g$ is in the control group, then $S_g^{(c)} = \overline{S}_g$. If $g$ is in the treatment group, then $S_g^{(t)} = \overline{S}_g + \Delta S_g$, where $\Delta S_g = r \cdot \overline{S}_g$. Moreover, the total increase in the spend satisfies the fixed budget assumption: $r\cdot\sum_{g\in\mathcal{T}}\overline{S}_g=B.$
\end{assumption}
\noindent Under this assumption, since we always increase the treatment spend proportionally to $\overline{S}_g$, we have
\begin{equation}\label{eq:theta_hetero_simplify}
\theta = \frac{\sum_g \theta_g \cdot \overline{S}_g}{\sum_g \overline{S}_g}.
\end{equation}
If we conduct the experiment and compute the empirical estimator $\hat{\theta}$ in the same way as before, it is no longer unbiased for $\theta$. Using the linear model assumption and the fixed budget assumption, we have
\begin{align}\label{eq:hetero_error}
    \hat{\theta} 
    %=&~\frac{\sum_{g \in \mathcal{T}} (Z_g + \theta_g (\overline{S}_g + \Delta S_g)) - \sum_{g \in \mathcal{C}} (Z_{g} + \theta_{g} \overline{S}_{g})}{\sum_{g \in \mathcal{T}} (\overline{S}_g + \Delta S_g) - \sum_{g \in \mathcal{C}} \overline{S}_{g}} \notag \\
    %= &~ \theta + \frac{1}{B} \cdot \sum_{g \in \mathcal{T}} (Z_g + (\theta_g - \theta) \cdot (\overline{S}_g + \Delta S_g)) \notag \\ 
    %&~ - \frac{1}{B} \cdot \sum_{g \in \mathcal{C}} (Z_{g} + (\theta_{g} - \theta) \cdot \overline{S}_{g}) \notag \\
    = \theta & + \frac{1}{B} \sum_{k} A_k (Z_{G_{k,+}} - Z_{G_{k,-}}) \notag \\
    & + \frac{1}{B} \Big( \sum_{g \in \mathcal{T}} (\theta_g - \theta) \overline{S}_g - \sum_{g \in \mathcal{C}} (\theta_{g} - \theta) \overline{S}_{g} \Big) \notag \\
    & + \frac{1}{B} \sum_{g \in \mathcal{T}} (\theta_g - \theta) \Delta S_g.
\end{align}
%where the second equality follows from the fixed budget assumption and the third follows from the definition of $A_k \in \{\pm 1\}$.
%\Shunhua{I shortened this equation here.}

\noindent We denote the three error terms in the above equation as $err_1$, $err_2$, and $err_3$. %Next we analyze them one by one, and give an explanation why sometimes it's not a good idea to discard data as in the trimmed match estimator.

%\begin{itemize}
%\item 
$err_1$: The first error term corresponds to the training error which is the same as in the homogeneous model. Since each $A_k$ is uniformly random in $\{-1,1\}$,
\[
\E[err_1] = \E\Big[\sum_{k} A_k (Z_{G_{k,+}} - Z_{G_{k,-}})\Big] = 0.
\]
The magnitude of this error term is small if the (super)geo pairs are well-matched. It can be reduced further by trimming some of the poorly matched pairs. 

%\item 
$err_2$: The second error term measures the difference between the treatment and control groups in terms of the within-group heterogeneity in $\theta_g$'s. Since each geo is either in the treatment group or the control group with probability $1/2$, we have
\begin{align*}
&~ \E[err_2]
= \E\Big[ \sum_{g \in \mathcal{T}} (\theta_g - \theta) \overline{S}_g - \sum_{g \in \mathcal{C}} (\theta_{g} - \theta) \overline{S}_{g} \Big] \\
= &~ \sum_{g \in \mathcal{G}} \E\Big[\mathbf{1}_{g \in \mathcal{T}} \cdot (\theta_g - \theta) \overline{S}_g - \mathbf{1}_{g \in \mathcal{C}} \cdot (\theta_g - \theta) \overline{S}_g \Big]
= 0.
\end{align*}
The second error term has zero mean for both the supergeo design and the matched pairs design. It remains zero when either all pairs are used or some are removed to improve the matching quality.% the empirical estimator or the trimmed match estimator. The trimmed match estimator could potentially decrease the magnitude of this error term since it removes some geos. %While for the supergeo design, since we group geos into supergeos according to $Z_g$'s which are independent of $(\theta_g - \theta) \overline{S}_g$, it is not helpful to reduce the magnitude of this error term.

%\item 
$err_3$: The third error term measures the degree of heterogeneity in $\theta_g$'s within the treatment group. The expectation of this term is approximately zero when all geos are included in the experiment. However, this term could induce a bias if some geos are excluded.
%We are going to show that the expectation of this term can be reduced by including all geos in the experiment. However, this term could induce a large bias if some geos are excluded.
%\Shunhua{I rephrased this sentence a bit.}

By Assumption \ref{ass:fixed_budget}, %$\Delta S_g \propto \overline{S}_g$, 
we have
\begin{equation*}%\label{eq:err3_simplify}
err_3 = \sum_{g \in \mathcal{T}} (\theta_g - \theta) \cdot \Delta S_g = r \cdot \sum_{g \in \mathcal{T}} (\theta_g - \theta) \cdot \overline{S}_g,
\end{equation*}
where $r$ is a random variable that depends on the specific treatment assignment in order to satisfy the fixed budget assumption.%\footnote{It has to vary between treatment assignments in order to satisfy the fixed budget assumption.}\Shunhua{I suggest to put this footnote in the main text.} \Aiyou{we already say it depends on the specific treatment assignment, so the footnote looks more like a duplicate? If so, could remove.}
%Since we also made the fixed budget assumption that $\sum_{g \in \mathcal{T}} \Delta S_g = B$, we have $r = \frac{B}{\sum_{g \in \mathcal{T}} \overline{S}_g}$.
%\begin{itemize}
%\item 

\paragraph{If all geos are included in the experiment.} The key observation is that since $\theta = \sum_g \theta_g \cdot \overline{S}_g / \sum_g \overline{S}_g$,
\begin{equation}\label{eq:hetero}\vspace{-0.2cm}
\sum_g (\theta_g - \theta) \cdot \overline{S}_g = 0.
\end{equation}
Since all geos are included in the experiment, $\sum_{g \in \mathcal{T}} \overline{S}_g \approx 0.5 \cdot \sum_{g \in \mathcal{G}} \overline{S}_g$, so that $r \approx r_0 := 2B/\sum_g \overline{S}_g$. Thus,
\begin{align*}
\E[err_3] = &~ \E\Big[ r \cdot \sum_{g \in \mathcal{T}} (\theta_g - \theta) \cdot \overline{S}_g \Big] \\
\approx &~ r_0 \cdot \E\Big[ \sum_{g \in \mathcal{T}} (\theta_g - \theta) \cdot \overline{S}_g \Big] \\
= &~ r_0 \cdot \sum_g \E[\mathbf{1}_{g \in \mathcal{T}} 
\cdot (\theta_g - \theta) \cdot \overline{S}_g] = 0,
\end{align*}
where the last equality follows from \eqref{eq:hetero} and the fact that each geo has probability $1/2$ to fall into the treatment group.

%\item 
\paragraph{If some geos are excluded.} In this case, some geos have zero probability to be included in the treatment group and we no longer have $\E[\sum_{g \in \mathcal{T}} (\theta_g - \theta) \cdot \overline{S}_g] = 0$. The bias induced by $err_3$ is especially large when the excluded geos have extra large or extra small values of $\theta_g$ since those differ the most from the weighted average value, $\theta$. %ese geos contribute a lot to the total sum. And this is exactly the case of the trimmed match estimator where it often excludes large geos.%$(\theta_g - \theta) \cdot \overline{S}_g$
%\end{itemize}
%\end{itemize}

While trimming poorly matched pairs can reduce the variance of the resulting estimator, it can also induce a substantial bias when $\theta_g$'s of the removed geos differ from the quantity of interest, $\theta$. We provide specific examples illustrating this issue in Section~\ref{sec:eval}. %In conclusion, while we can improve the vanilla matched pairs design by using either the supergeo design or the trimmed match estimator, it may not be a good idea to trim pairs when the $(\theta_g - \theta) \cdot \overline{S}_g$ values are very non-uniform.
%would greatly affect the third error term if it excludes geos whose $(\theta_g - \theta) \cdot \overline{S}_g$ contributes a lot to the total sum.
%\Shunhua{Please polish this sentence. We may not want to use trimmed match when some $(\theta_g - \theta) \cdot \overline{S}_g$ are outliers, and TM trims them.}
\section{Algorithm}\label{sec:supergeo_algo}
%In this section we present algorithms to find the optimal supergeo design in practice.

%\subsection{Supergeo design in practice}\label{sec:problem_formulation}
Following our theoretical model from Section~\ref{sec:model}, the goal is to find a supergeo design $\{(G_{k,+}, G_{k,-})\}_{k=1}^K$ that minimizes the following loss:
\begin{equation}\label{eq:loss}
\mathrm{loss}(\{(G_{k,+}, G_{k,-})\}_{k=1}^K) := \sum_{k=1}^K (Z_{G_{k,+}} - Z_{G_{k,-}})^2.
\end{equation}
This loss coincides with the variance of the empirical estimator in the homogeneous $\theta_g$ case (Eq.~\eqref{eq:variance}) and corresponds to the variance of term $err_1$ in the heterogeneous $\theta_g$ case. This term is likely to dominate the other two, especially in online advertising applications where ad spend is often at a much smaller scale than revenue.
%is essentially the only term we can optimize since the other two error terms both depend on $\theta_g$'s that are completely unknown.

%the optimal supergeo design is the design $\{(G_{k,+}, G_{k,-})\}_{k=1}^K$ that minimizes the error of $\hat{\theta}$. In the homogeneous $\theta_g$ model, $\hat{\theta}$ is an unbiased estimator of $\theta$ with variance shown in Eq.~\eqref{eq:variance}, so we can find the optimal design by minimizing the loss 

%\Shunhua{TODO: justify why using this loss is good for both homogeneous and heterogeneous models.}
%In the heterogeneous $\theta_g$ model, this loss corresponds to the first error term in Eq.~\eqref{eq:hetero_error}, and it's still a natural choice since the other two error terms all depend on $\theta_g$'s that are completely unknown.
\paragraph{Matching variables.}
While ideally we would want to match supergeo pairs based on the values of uninfluenced responses, $Z_g$, those are usually unobserved. We follow the existing literature \citep{chen2021trimmed} and approximate the uninfluenced responses by the responses in the pretest phase. Throughout this section we continue using $Z_g$'s to denote the variables we match on, but in practice pretest responses are often used.

%This is often a good approximation since the control variables, $S_g$, are usually much smaller than the responses $R_g$'s. 

\paragraph{Size of supergeo pairs.}
%\Shunhua{To discuss: should we move this "subset size" paragraph to the previous "model" section? Or move the whole Section 3.1 to the "model" section?}
As already discussed, the loss in Eq.~\eqref{eq:loss} is minimized by dividing all geos into two supergeos, i.e.~when $K=1$. However, in practice this is often not desirable for a variety of reasons:
%We further add a restriction that the size of each supergeo pair is in some range $[\ell, u]$, i.e.~$\ell \leq |G_{k,+}| + |G_{k,-}| \leq u$ for all $k$, and $\ell$ and $u$ are two tunable parameters. We usually set $\ell = 2$ so that the supergeo design is a natural generalization of the matched pairs design. While a larger $u$ always leads to a smaller loss, in practice we prefer to choose a smaller $u$ for the following reasons:
\begin{itemize}
\item {\bf Overfitting.} Since the observed response variables do not have to follow our linear model and since they approximate the true uninfluenced responses which are unobserved,  %\Shunhua{Should we put the "pretest responses" paragraph before the "Size of supergeo pairs" paragraph? Here we refer to the error induced by approximating $Z_g$'s using the pretest responses.}\Aiyou{+1} 
using large supergeos could lead to overfitting to the pretest data. In our experiments we observe that while increasing the maximum allowed size of supergeos leads to smaller training losses, the performance on the test set may suffer (see Appendix~\ref{sec:appendix-B-eval} for an example). This issue can be resolved by tuning the maximum allowed size of supergeo pairs---splitting the pretest data into two subsets, one for creating candidate designs and the other for validation---in the spirit of cross-validation used in machine learning. %\Shunhua{Should we put the last sentence to the future work section? Or explain somewhere that currently we are not using this cross-validation.} %, in the same spirit as cross validation, which is common for machine learning, but relatively new for experimental designs \citep[see][]{chen2021trimmed}. %\Shunhua{Add a remark about cross validation.}\Aiyou{Added a sentence.}
\item {\bf Inference.} Using fewer supergeo pairs makes statistical inference harder. For example,
the trimmed match estimator from \citet{chen2022robust} uses $t$-distribution as an approximation to construct the confidence intervals. This approximation requires at least two pairs and the quality of the approximation improves as the number of pairs grows. Similarly, when using inference approaches such as Fisher's exact test \citep{fisher1922interpretation}, we want to allow for many possible treatment assignments as opposed to just two which corresponds to the case of two large supergeos (see Appendix~\ref{sec:inference}).
\item {\bf Robustness.} Increasing the number of supergeo pairs increases the number of possible treatment assignments which makes the experimental design more robust to adversarial considerations (see Appendix~\ref{sec:size} for details). 
\item {\bf Computational efficiency.} Increasing the maximum allowed size of supergeo pairs increases the number of candidate pairs which could be included in the optimal design. Depending on the formulation of the optimization problem, this often leads to the problem being intractable. %too slow to solve. %intractable.
\end{itemize}

\noindent For these reasons, we add a constraint that the size of each supergeo pair is within a range, $[\ell, u]$, i.e.~$\ell \leq |G_{k,+}| + |G_{k,-}| \leq u$ for all $k$, where $\ell$ and $u$ are two tunable parameters. We usually set $\ell = 2$ so that the supergeo design is a natural generalization of the matched pairs design. When reporting the empirical results in Section~\ref{sec:eval}, we limit the size of supergeo pairs to 4 which performs well in our applications.\footnote{We currently do not tune this parameter and set it to a fixed value that provides a good matching quality without leading to particularly large supergeos.}

The first three of the concerns above could also be addressed by adding an explicit ``minimum pairs'' constraint that restricts the number of supergeo pairs to be at least $\kappa$. However, this constraint is ineffective in addressing the issue of computational efficiency. Since the size constraint (combined with the heuristics that we use for the MIP formulation discussed in Appendix~\ref{sec:heuristics}) leads to a design with a sufficient number of supergeo pairs, in our empirical evaluations we set $\kappa = 1$.

%Note that $u$ prevents us from aggregating too many geos into a single supergeo. On the other hand, it could also prevent us from creating singleton-supergeo pairs, even if they are well-matched. For example, with 40 very similar geos, where the optimal design would make 20 pairs, but with $u=4$ we would only get 10 supergeo pairs, which is less desirable. This may lead to another source of overfitting. To resolve this, we add another tuning parameter $\kappa$ about the minimum number of supergeo pairs, i.e. $K\geq \kappa$. $\kappa$ can be chosen by cross validation as mentioned earlier. The per-geo heuristic that we will introduce in Section~\ref{sec:heuristics} is also helpful in preserving the singleton-supergeo pairs.\Aiyou{Nick please take a look/edit this new paragraph.}\Nick{To be honest, I would currently comment this out. It's not quite how we use $u$---setting it at 4 does not prevent us from considering smaller supergeo pairs.}\Aiyou{It does not prevent smaller supergeos, but prevent supergeos with a single geo. Am I right?}

%We also add a minimum number of supergeo pairs constraint to that could further restrict the design to have many supergeo pairs. 
%\Shunhua{@Aiyou, I added a sentence here. I'm intentionally being vague here since in evaluations we set the minimum number of supergeo pairs to be 1, so it's not actually restricting anything currently. But we have the option to make use of this constraint by setting a larger number.}\Aiyou{@Shunhua, I drafted a few sentences from the statistical perspective. Could you revise? I think it is okay to tell the specific value of $\kappa$ for the experiments.}\Shunhua{It looks good to me. I added that the per-geo heuristic is also helpful.}\Aiyou{LGTM} %, and we choose $u$ to be in $\{3,4\}$.
%\Cliff{Do we want to say that computational tractability is a reason too?}
%\Aiyou{Since cross validation is not used, what criteria do we use to decide which size to use in the end?}
%\Shunhua{We actually fixed the size to be 4 in the experiments. We could change the size, and pick the design with the smallest training loss.}
%\Aiyou{I see. It looks that the simulation already uses a test set, so we could just use that test results to choose the size. Then for actual evaluation, simulate another test set? The new test results would be quite similar to the ones reported here for the same size, but makes the presentation stronger.}
%\Aiyou{Per discussion, minimization of training loss always leads to the largest u; Leaving it to future work is a good idea. Shall we add that we fix the size to be 4 for the numerical studies?}



\subsection{NP-hardness}
For a supergeo pair $(G_{k,+}, G_{k,-})$, define $G_{k} := G_{k,+} \cup G_{k,-}$. Observe that in order to minimize the loss~\eqref{eq:loss}, $(G_{k,+}, G_{k,-})$ must be an optimal split that minimizes $(Z_{G_{k,+}} - Z_{G_{k,-}})^2$ among all splits of $G_k$:
\[
(G_{k,+}, G_{k,-}) \in {\arg \min}_{\substack{G'_{k,+} \cup G'_{k,-} = G_k \\ G'_{k,+} \cap G'_{k,-} = \emptyset}} (Z_{G'_{k,+}} - Z_{G'_{k,-}})^2.
\]
While there may be multiple optimal splits, $(G_{k,+}, G_{k,-})$, the minimum possible distance between them is uniquely determined given the union $G_k$. Consequently, we can define the loss~\eqref{eq:loss} with respect to $G_k$'s. For any subset $G \in \mathcal{G}$, let
\[
\mathrm{score}(G) := \min_{\substack{G_+ \cup G_- = G \\ G_+ \cap G_- = \emptyset}} (Z_{G_+} - Z_{G_-})^2.
\]
Our goal is to find a partition of $\mathcal{G}$ into disjoint subsets $\{G_k\}_{k=1}^K$ that minimize the loss
\[
\mathrm{loss}(\{G_k\}_{k=1}^K) := \sum_{k=1}^K \mathrm{score}(G_k).
\]
%\Shunhua{To discuss: is the current notation $G_k$, $G_1, G_2$, $G_{k,+}, G_{k,-}$ confusing? Change $G_{k,+}, G_{k,-}$ to $G_{k,+}, G_{k,-}$?}\Bicheng{+1 on $G_{k,+}, G_{k,-}$}









%The variance in Eq.~\eqref{eq:variance} implies that in order to design a ``good'' experiment we want to set the difference of baseline responses within the same group to be as small as possible. In the geo matching design, this is equivalent to a minimum-weight matching problem, while in the supergeo matching design, this is a generalized hyper-graph matching problem. Since the magnitude of the control variable $S_g$ is usually much smaller than that of the responses $R_g$, we approximate $Z_g$ by the response value of the pretest data.

% \Shunhua{TODO: mention the problem is NP hard, and point to appendix}
% \Nick{I would also use NP-hardness as a motivation to use MIP.}



\noindent While the optimal matched pairs design can be found in polynomial time, the proposed generalized matching problem is NP-hard even when the size of $G_{k}$ increases from 2 to 3 (setting $\ell = u = 3$).
\begin{theorem}[NP-hardness]\label{thm:np_hard}
The following problem is NP-hard: Given a set $\mathcal{G}$ of size $|\mathcal{G}|=3m$ and values $Z_g \in \mathbb{Z}^+$ for each $g \in \mathcal{G}$, for any subset $G \subseteq [3m]$ define
\[
\mathrm{score}(G) := \min_{\substack{G_+ \cup G_- = G \\ G_+ \cap G_- = \emptyset}} \Big( \sum_{i \in G_+} Z_{i} - \sum_{j \in G_-} Z_{j} \Big)^2.
\]
Determine whether $\mathcal{G}$ can be partitioned into $m$ disjoint sets $G_1, G_2, \dots, G_m$ such that for all $i \in [m]$, $|G_i|=3$ and $\mathrm{score}(G_i) = 0$ ($[m]$ stands for the set $\{1,2,\dots,m\}$).
\end{theorem}
\noindent Note that this decision problem is easier than the problem we are interested in (loss minimization).
%\Aiyou{It is unlikely to get $\mathrm{score}(G_i) = 0$. Is the idea to say that the actual minimization problem is harder than this?}\Shunhua{I added an explanation.}
We include a formal proof of the theorem in Appendix~\ref{sec:np_hard}.

\subsection{MIP formulation}\label{sec:mip}
%\Cliff{I think that the two papers we cite use a very different formulation.  We should empasize that we are using a covering formulation and that this is important}\Shunhua{Done.}
To tackle this NP-hard problem, we propose a covering formulation and solve it using mixed-integer programming (MIP). The idea of using MIP for experimental design is not new \citep[see, for example,][]{zubizarreta2012using,doudchenko2021synthetic,abadie2021synthetic}, but our covering formulation is novel.
%Similar to \citet{doudchenko2021synthetic} and \citet{abadie2021synthetic}, we formulate the problem as a mixed-integer program (MIP). 

Define $\mathcal{F} := \{G \subseteq \mathcal{G} \mid \ell \leq |G| \leq u\}$. Our variable is a boolean vector $x \in \{0,1\}^{|\mathcal{F}|}$ such that component $x_G$ indicates whether a candidate supergeo pair $G$ is included in the supergeo design. 
%For any $G \subseteq \mathcal{G}$ with size $\ell \leq |G| \leq u$, we use a binary variable $x_G \in \{0,1\}$ to indicate whether $G$ is included in the supergeo design.
For each geo $g$, we add a linear constraint that restricts $g$ to be covered by exactly one supergeo pair. To express this constaint, we define a vector $M^{(g)} \in \{0,1\}^{|\mathcal{F}|}$ such that $M^{(g)}_{G} = 1$ if $g \in G$ and $M^{(g)}_{G} = 0$ otherwise. The constraint that geo $g$ is covered by exactly one supergeo pair is equivalent to $(M^{(g)})^{\top} x = 1$. We stack these column-vectors into a matrix $M := [M^{(g_1)} \cdots M^{(g_N)}] \in \{0,1\}^{|\mathcal{F}| \times N}$. 
%Often, we don't want to merge all geos into too few supergeos. To avoid this, researcher can set a scalar $\kappa$, the smallest number of groups, and add the constraint to force the design to generate enough supergeos.

Let $\mathds{1}_d$ denote the all-one vector of dimension $d$ for any $d$. The mixed-integer program we solve is as follows:
\begin{align*}
    \min_{x}&\;\; \sum_{G \subseteq \mathcal{G}:\ \ell \leq |G| \leq u} \mathrm{score}(G) \cdot x_G \\
    \text{s.t.}&
    ~~~~M^{\top} x = \mathds{1}_N,  \hspace{1.5cm} & \mbox{(Exact cover)}\\
    &~~~~\mathds{1}_{|\mathcal{F}|}^\top \cdot x \geq \; \kappa, &\mbox{(Minimum pairs)} \\
    &~~~~ x \in \{0,1\}^{|\mathcal{F}|}. & \mbox{(Boolean selection)}
%     \begin{aligned}
%     M^{\top} x =&\; \mathds{1}_n,  \hspace{1.5cm}&\mbox{(Exact cover)}\\
% %     \mathds{1}_{m}^T x \geq&\; \kappa, &\mbox{(Minimum pairs)}\\
%      x \in\;& \{0,1\}^{|\mathcal{G}|}. &\mbox{(Boolean selection)}
%     \end{aligned}
\end{align*}
% \begin{align*}
%     M:= \begin{bmatrix}
%     M_{g_1}^{\top} \\
%     \vdots \\
%     M_{g_n}^{\top}
%     \end{bmatrix}, \;\;\;\mathds{1} := \begin{bmatrix}
%     1 \\
%     \vdots \\
%     1
%     \end{bmatrix}.
% \end{align*}
%\Aiyou{Shall we add the number of pairs as another tuning parameter? This looks important intuitively -- otherwise this would rule out singleton supergeos, as two singleton-supergeo pairs would always be aggregated into one supergeo pair, which may lead to overfitting. BTW this may provide another reason (probably more reasonable than a computational excuse :) about why there are many singleton supergeo pairs in Figure 3.}\Shunhua{We are not using the number of pairs constraint in our experiments. In practice we observed that sometimes when adding the number of pairs constraint the MIP becomes harder to solve.}\Bicheng{I checked the code, we do have $\mathds{1}^{\top} x >= 1$ constraint in the experiment.}\Aiyou{Glad the code takes care of that. From statistical perspective, u makes sure we don't produce elephant supergeos, which is critical, but it also rules out well-matched supergeo pairs which could be small; So adding the tuning parameter on number of pairs will help us preserve the small well-matched supergeo pairs. I can imagine an ideal case with 40 very similar geos, for which 20 pairs is optimal, but our current formulation would lead to only 10 pairs. How many pairs ended up for the experiments?}\Shunhua{For 210 geos we had $\approx 80$ pairs. I've added back the number of pairs constraint.}\Aiyou{+1 maybe also add a discussion next to 'size of supergeo pairs'?}
We use SCIP solver of \citet{BestuzhevaEtal2021OO} to solve this MIP which can be done under 1 hour for approximately $N=50$ geos and the maximum size of supergeo pairs equal to 4. To speed up the optimization and make the problem tractable at the practically important scale ($N=210$) we employ a number of heuristics discussed in Appendix~\ref{sec:heuristics}.\footnote{The most straightforward MIP formulation is to use $O(N^2)$ variables such that variable $(i,j)$ indicates whether the $i$-th geo is in the $j$-th supergeo. Our covering formulation has $O(N^u)$ variables. However, it is much easier to solve due to the numerous symmetries between variables intrinsic to the alternative formulation.}
%\Shunhua{Please check this sentence.}
%\Aiyou{Can we add an example with roughly how much faster our covering MIP is than the straight-forward one?}\Shunhua{Even in the small case with $N=15$ geos and $\kappa =4$ supergeo pairs, I wasn't able to solve the straight-forward MIP within an hour. Given this, I think it's fair to say the straight-forward MIP would take "forever" on 210 geos. We gave up this formulation pretty quickly. I had some experimental codes in \url{https://source.corp.google.com/piper///depot/google3/experimental/users/shunhuajiang/supergeos/} that you are welcomed to try out.}\Aiyou{I see. LGTM then.}




\section{Empirical results}\label{sec:eval}
We evaluate the proposed design in the context of estimating the incremental return on ad spend (iROAS), where for each geographic region the ad spend and the response (e.g.~conversions or sales) are observed. For each geo $g \in \mathcal{G}$, we collect the spend $S_g[1],\dots,S_g[T]$ and responses $R_g[1],\dots,R_g[T]$ over $T$ time periods. We divide $T$ into a pretest phase $\{1,\dots,T_0\}$ and a test phase $\{T_0+1,\dots,T\}$, and refer to the aggregates $R_g^{\mathrm{pre}} := \sum_{t=1}^{T_0} R_g[t]$ and $R_g^{\mathrm{test}} := \sum_{t=T_0 + 1}^{T} R_g[t]$ as the (total) pretest and test responses respectively. We define the pretest spend $S_g^{\mathrm{pre}}$ and the test spend $S_g^{\mathrm{test}}$ similarly. We use the pretest response $R_g^{\mathrm{pre}}$ as the approximation of $Z_g$ which is unobserved and we use the algorithms described in Section~\ref{sec:supergeo_algo} to generate experimental designs. The tuning parameters $\ell$ (minimum size of a supergeo pair), $u$ (maximum size of a supergeo pair), and $\kappa$ (minimum number of supergeo pairs) are fixed at 2, 4, and 1 respectively.%\Aiyou{Shall we add that for computational reasons, the tuning parameters $u$ and $\kappa$ are fixed instead of cross validation}

We iterate over different treatment assignments performing what we call the ``half-synthetic'' evaluations---the underlying data comes from real online advertising settings with artificial treatment ``injected'' in the data. Specifically, at each iteration we increase (\emph{``heavy-up''}) the spend in each treated geo in the test period to $(1 + r) \cdot S_g^{\mathrm{test}}$ and leave the spend in the control geos unaffected.\footnote{The methods of this paper can also be applied to another commonly used type of experiment in which the spend in treated geos is set to zero (\emph{``go dark''}).} The response variable in a treated geo $g$ is increased to $R_g^{\mathrm{test}} + \theta_g \cdot r S_g^{\mathrm{test}}$, where $\theta_g$ is the iROAS for that geo that is chosen by us and depends on the particular evaluation as discussed further. The ratio $r$ is set in such a way that the total increase in the spend is equal to the fixed budget, $B$.

The underlying data comes from one of the two real datasets that we call ``A'' and ``B.'' In both datasets $\mathcal{G}$ corresponds to the set of 210 DMAs in the United States, and the unit of time is one day. We set both the pretest and test phases to be four weeks long. The budget, $B$, is chosen to roughly match the magnitude of the total spend in the absence of treatment effects. %\Shunhua{Check the total spends.}
The true response and spend values are scaled to be between $[0,1]$. For all experiments reported in this paper, the evaluation procedures are iterated $M=10000$ times across different treatment assignments. %To generate the supergeo design, we set the maximum supergeo pair size ($u$) to be 4 and minimum number of pairs ($\kappa$) to 80.\Aiyou{Shunhua or Nick: is kappa equal to 80?}
We run the heuristics from Appendix~\ref{sec:heuristics} with different parameters in parallel with a time limit of 3 hours, and then pick the design that leads to the smallest training loss. 

We compare the supergeo design to matched pairs design using the empirical estimator~(Eq.~\eqref{eq:theta_hat}) as well as the trimmed match estimator~(Eq.~\eqref{eq:theta_hat_trim}). The poorly matched pairs can be trimmed during either the design stage or the analysis stage (or both). This has the potential to greatly reduce the variance of the resulting estimator, but may introduce substantial bias if the trimmed geos differ from the remaining ones in terms of their response to treatment. In the evaluations we have conducted, the two approaches---trimming at the design stage or at the analysis stage---led to similar results with the latter being generally more effective since it can take into account not only the similarity of geos in the pretest period, but also their potential divergence in the test period due to noise. Consequently, we present the comparisons between the supergeo design which utilizes a simple empirical estimator and the matched pairs design which utilizes the trimmed match estimator from \citet{chen2022robust}. %which implements trimming at the analysis stage. 
However, the takeaways from the analysis would remain similar if the trimming was done at the design phase. We also emphasize that while the main advantage of the supergeo design is the ability to improve the matching quality without sacrificing any data, this design can be combined with trimming. This leads to a performance that is similar or better than that of the trimmed match estimator applied to matched pairs design, but may still lead to a bias when the effects are heterogeneous. 

\subsection{Homogeneous $\theta_g$}\label{sec:evaluation_homogeneous}
First, we present the comparisons between the different methodologies for the case of homogeneous $\theta_g=1$ for all $g\in\mathcal{G}$. In Table~\ref{tab:homogeneous} we report the \emph{root-mean-square error} (RMSE), $\sqrt{M^{-1}\sum_{m=1}^M (\hat{\theta}^{(m)} - \theta)^2}$, and the absolute bias, $|M^{-1}\sum_{m=1}^M \hat{\theta}^{(m)} - \theta|$, where $\hat{\theta}^{(m)}$ corresponds to the estimate computed in iteration $m$. %shows the RMSE and the bias of the matched pairs design and the supergeo design using both the empirical estimator ($\hat{\theta}$ as defined in Eq.~\eqref{eq:theta_hat}) and the trimmed match estimator ($\hat{\theta}^{\mathrm{trim}}$ as defined in Eq.~\eqref{eq:theta_hat_trim}). 
%$\hat{\theta}$ and the coverage of nominally 80\% confidence intervals, where the coverage is computed as the percentage of the estimates that fall into the computed confidence intervals.
See Figure~\ref{fig:homogeneous} in Appendix~\ref{sec:other_figs} for the histograms of the estimates.


% \begin{table}[!h]
% \centering
% \begin{tabular}{l|l|cc|cc}
%     \toprule
%     \multirow{2}{*}{Est.} & \multirow{2}{*}{Design} & \multicolumn{2}{c|}{Dataset A} & \multicolumn{2}{c}{Dataset B} \\
%     \cmidrule(r){3-6}
%     & & RMSE & Cov. & RMSE & Cov. \\
%     \midrule
%     \multirow{2}{*}{$\hat{\theta}$} & Pairs & $0.97$ & $74\%$ & $2.03$ & $77\%$ \\
%      & Supergeo & $0.42$ & $79\%$ & $0.30$ & $78\%$ \\
%     \midrule
%     \multirow{2}{*}{$\hat{\theta}^{\mathrm{trim}}$} & Pairs & $0.33$ & $76\%$ & $0.42$ & $79\%$ \\
%      & Supergeo & $0.34$ & $78\%$ & $0.37$ & $72\%$ \\
%     \bottomrule
%   \end{tabular}
% \caption{Evaluation data under homogeneous iROAS. Root-mean-square errors of the estimates $\hat{\theta}$ and the coverage (abbreviated as Cov.) of nominally 80\% confidence intervals. The estimator (abbreviated as Est.) is either the empirical estimator $\hat{\theta}$ or the trimmed match estimator $\hat{\theta}^{\mathrm{trim}}$, and the experimental design is either the matched pairs design (abbreviated as Pairs) or the supergeo design (abbreviated as Supergeo).}
% \label{tab:homogeneous}
% \end{table}
\begin{table}[!h]
\centering
\begin{tabular}{l|l|cc|cc}
    \toprule
    \multirow{2}{*}{Est.} & \multirow{2}{*}{Design} & \multicolumn{2}{c|}{Dataset A} & \multicolumn{2}{c}{Dataset B} \\
    \cmidrule(r){3-6}
    & & RMSE & Bias & RMSE & Bias \\
    \midrule
    \multirow{2}{*}{$\hat{\theta}$} & Pairs & $0.97$ & $0.022$ & $2.03$ & $0.107$ \\
     & Supergeo & $0.42$ & $0.005$ & $0.30$ & $0.006$ \\
    \midrule
    \multirow{2}{*}{$\hat{\theta}^{\mathrm{trim}}$} & Pairs & $0.33$ & $0.006$ & $0.42$ & $0.003$ \\
     & Supergeo & $0.34$ & $0.006$ & $0.37$ & $0.016$ \\
    \bottomrule
  \end{tabular}
  
\medskip
\raggedright
{\small\textit{Note}: The estimator (abbreviated as Est.) is either the empirical estimator $\hat{\theta}$ or the trimmed match estimator $\hat{\theta}^{\mathrm{trim}}$. The experimental design is either the matched pairs design (abbreviated as Pairs) %\Aiyou{We compare standard geo-based matched pairs and supergeo pairs, so use 'Geo' or  or 'Standard' (standard matched pairs) instead of 'Pairs'?}
or the supergeo design (abbreviated as Supergeo).}
\caption{Empirical results under homogeneous iROAS: the RMSE and the bias of the estimates.}
\label{tab:homogeneous}
\end{table}


As evident from the table, supergeo design almost uniformly outperforms the matched pairs design in terms of RMSE when both designs use the same estimator.\footnote{The only exception being the trimmed match estimator on dataset A, where the supergeo design is very slightly worse than the matched pairs design.} Moreover, the performance of the empirical estimator applied to supergeo design is comparable to that of the trimmed match estimator applied to the matched pairs design (supergeo design is better on dataset B, and worse on dataset A)---supergeo design is able to achieve similar performance without throwing out any data.
%\Cliff{Shouldn't we say that without trimming, our RMSE is way better than paired.  Trimming doesn't seem to help us much and even sometimes it hurts.  So we are able to get similar performance to trimming with pairs, but we are not throwing out any of the data (and we gave the negatives of throwing out data earlier).}

In Figure~\ref{fig:design_small} we report the matching quality achieved by the two designs in the pretest period. %\footnote{Note that there are still many one-to-one pairs in the supergeo design. This is the result of using the per-geo heuristic from Section~\ref{sec:heuristics} which aims to include the well-matched supergeo pairs containing some of the largest geos. The smaller geos are already matched well in one-to-one pairs.}\Aiyou{The footnote looks no longer needed if we keep $\kappa$ to permit singleton supergeos?}
We can see that supergeo design achieves a much better matching quality than the matched pairs design on the pretest data. Figure~\ref{fig:design} in Appendix~\ref{sec:other_figs} shows that on the test data, the matching quality of the supergeo design declines relative to the pretest period due to temporal noise. We also observe that this noise is more substantial in dataset A compared to dataset B, and we conjecture that this is the reason behind supergeo design performing worse on dataset A.

\begin{figure}[!tb]
\centering
\subfigure[Matched pairs design]{\label{fig:A_baseline_design_small}{\includegraphics[trim={0 18cm 2.5cm 3.9cm},clip,width=0.48\textwidth]{figs/old_pilot_baseline/design_plot.pdf}}}
\subfigure[Supergeo design]{\label{fig:A_supergeo_design_small}{\includegraphics[trim={0 18cm 2.5cm 3.9cm},clip,width=0.48\textwidth]{figs/old_pilot/design_plot.pdf}}}
%\vspace{-0.4cm}

\medskip
\raggedright
{\small\textit{Note}: Each dot represents one (super)geo and the dashed line between two dots implies that these two (super)geos are matched with ``left'' and ``right'' representing the two ``sides'' of each (super)geo pair. The $x$-axis is the scaled value of total pretest response. The blue circle represents a single geo, the orange cross represents a supergeo consisting of 2 geos, the green square represents a supergeo consisting of 3 geos.}
\caption{Matching quality achieved by different designs on dataset A in the pretest phase.
}
\label{fig:design_small}
%\vspace{-0.2cm}
\end{figure}



\subsection{Heterogeneous $\theta_g$}
Second, we consider the case when $\theta_g$'s can vary across geos. We modify $\theta_g$ by adding a compotent that is proportional to the uninfluenced response of a geo: $\theta_g = 1 + c \cdot Z_g/(N^{-1}\sum_{g} Z_g)$, where $c=0.2$ when applied to dataset A and $0.07$ when applied to dataset B. Following Eq.~\eqref{eq:theta_hetero_simplify}, the quantity of interest is computed as $\theta = \sum_g\theta_g \cdot S_g^{\mathrm{test}}/\sum_g S_g^{\mathrm{test}}$. %\Nick{@Shunhua why $\leq$?}\Shunhua{Because for dataset A we choose $c=0.2$ and for dataset B we choose a smaller $c=0.07$ (since dataset B is a bit more noisy). We could also state these two numbers separately.} is a small constant.

This model captures a scenario in which advertising is more effective in densely populated areas that often generate larger revenues. This assumption may or may not be adequate in a particular setting. What is important, is the fact that trimming of the poorly matched pairs will likely be non-random with respect to the distribution of $\theta_g$'s across geos. In Appendix~\ref{sec:other_hetero} we report the results from another simulation study with heterogeneous $\theta_g$'s---each $\theta_g$ equals $1$ plus a uniformly random noise. Under this model, the results are similar to those from the homogeneous model---while the sample size is reduced, the geos are trimmed in a way that is random with respect to the distribution of $\theta_g$'s, resulting in no substantial bias. This confirms our intuition from Section~\ref{sec:hetero_model}. %\Shunhua{I think the message in this footnote is important and we should put it in the main text.} %\Shunhua{Please check this sentence. We need a good explanation of why this is a reasonable model.}

Table~\ref{tab:heterogeneous} reports the RMSE and the absolute bias while Figure~\ref{fig:heterogeneous} in Appendix~\ref{sec:other_figs} shows the histograms of the estimates. Supergeo design---when combined with the empirical estimator---reduces the variance without introducing a bias and leads to the most precise estimates across the board. The trimmed match estimator is effective at reducing the RMSE of the matched pairs design by reducing the variance, but that is achieved at the cost of introducing a bias which may be substantial. Moreover, it does not improve the results over the empirical estimator when the supergeo design is used---the potential reduction in variance is not worth the price paid in the increased bias.%The trimmed match estimator may already performs pretty good using the empirical estimator, trimmed match estimator is no longer helpful. Our supergeo design with the simple empirical estimator achieves better RMSE than the trimmed match estimator, and this is consistent with our theoretical model in Section~\ref{sec:hetero_model}.

% \begin{table}[!h]
% \centering
% \begin{tabular}{l|l|cc|cc}
%     \toprule
%     \multirow{2}{*}{Est.} & \multirow{2}{*}{Design} & \multicolumn{2}{c|}{Dataset A} & \multicolumn{2}{c}{Dataset B} \\
%     \cmidrule(r){3-6}
%     & & RMSE & Cov. & RMSE & Cov. \\
%     \midrule
%     \multirow{2}{*}{$\hat{\theta}$} & Pairs & $1.01$ & $75\%$ & $2.17$ & $77\%$ \\
%      & Supergeo & $0.38$ & $81\%$ & $0.47$ & $72\%$ \\
%     \midrule
%     \multirow{2}{*}{$\hat{\theta}^{\mathrm{trim}}$} & Pairs & $0.58$ & $47\%$ & $0.65$ & $49\%$ \\
%      & Supergeo & $0.54$ & $42\%$ & $0.52$ & $52\%$ \\
%     \bottomrule
%   \end{tabular}
% \caption{Evaluation data under heterogeneous iROAS, where the heterogeneous iROAS is proportional to the geo size. The table is shown in the same format as Table~\ref{tab:homogeneous}.}
% \label{tab:heterogeneous}
% \end{table}

\begin{table}[!h]
\centering
\begin{tabular}{l|l|cc|cc}
    \toprule
    \multirow{2}{*}{Est.} & \multirow{2}{*}{Design} & \multicolumn{2}{c|}{Dataset A} & \multicolumn{2}{c}{Dataset B} \\
    \cmidrule(r){3-6}
    & & RMSE & Bias & RMSE & Bias \\
    \midrule
    \multirow{2}{*}{$\hat{\theta}$} & Pairs & $1.01$ & $0.040$ & $2.17$ & $0.140$ \\
     & Supergeo & $0.38$ & $0.003$ & $0.47$ & $0.004$ \\
    \midrule
    \multirow{2}{*}{$\hat{\theta}^{\mathrm{trim}}$} & Pairs & $0.58$ & $0.429$ & $0.65$ & $0.507$ \\
     & Supergeo & $0.54$ & $0.442$ & $0.52$ & $0.420$ \\
    \bottomrule
  \end{tabular}
  
\medskip
\centering
{\small\textit{Note}: The table is shown using the same format as is used in Table~\ref{tab:homogeneous}.}
\caption{Empirical results under heterogeneous iROAS.}
\label{tab:heterogeneous}
\end{table}



\section{Future work}
There are a number of potential future extensions:
\begin{itemize}
  \item Running experiments on half of the entire population is expensive. If the treatment fraction could be reduced, the cost savings would often surpass the potential loss in statistical power. This can be achieved, for example, by constructing supergeo triplets, quadruplets, etc.~and assigning only one supergeo in each group to treatment. 
  %\item supergeo design to reduce variance of trimmed match estimator
  \item The current MIP formulation can be extended to include additional matching variables such as: socio-demographic characteristics, distance between geos, etc. Geo aggregation will need to be adjusted accordingly.
  \item The current design does not explicitly account for proximity of the regions, or travel patterns introducing additional interference/contamination. These considerations can be taken into account when constructing supergeos.
\end{itemize}

%We have proposed a new design methodology which generalizes the standard matched pairs design by allowing the flexibility to aggregate multiple experimental units into a super-unit. The new design is shown to significantly improve the matching quality for geographical experiments, which is nowadays the gold standard for online advertising. Our work naturally raises a few interesting open problems for future research:...
% \setlength{\abovedisplayskip}{5pt}
% \setlength{\belowdisplayskip}{5pt}

% In the unusual situation where you want a paper to appear in the
% references without citing it in the main text, use \nocite
%\nocite{langley00}

\clearpage
\newpage
\bibliography{ref}
\bibliographystyle{icml2023}


%%%%%%%%%%%%%%%%%%%%%%%%%%%%%%%%%%%%%%%%%%%%%%%%%%%%%%%%%%%%%%%%%%%%%%%%%%%%%%%
%%%%%%%%%%%%%%%%%%%%%%%%%%%%%%%%%%%%%%%%%%%%%%%%%%%%%%%%%%%%%%%%%%%%%%%%%%%%%%%
% APPENDIX
%%%%%%%%%%%%%%%%%%%%%%%%%%%%%%%%%%%%%%%%%%%%%%%%%%%%%%%%%%%%%%%%%%%%%%%%%%%%%%%
%%%%%%%%%%%%%%%%%%%%%%%%%%%%%%%%%%%%%%%%%%%%%%%%%%%%%%%%%%%%%%%%%%%%%%%%%%%%%%%
\newpage
\appendix
\onecolumn
\begin{figure*}[!ht]
\centering
\subfigure[Dataset A: Matched pairs design]{\label{fig:A_baseline_design}{\includegraphics[trim={0 2cm 2.5cm 2cm},clip,width=0.44\textwidth]{figs/old_pilot_baseline/design_plot.pdf}}}
\subfigure[Dataset A: Supergeo design]{\label{fig:A_supergeo_design}{\includegraphics[trim={0 2cm 2.5cm 2cm},clip,width=0.44\textwidth]{figs/old_pilot/design_plot.pdf}}}
\subfigure[Dataset B: Matched pairs design]{\label{fig:B_baseline_design}{\includegraphics[trim={0 2cm 2.5cm 2cm},clip,width=0.44\textwidth]{figs/new_pilot_baseline/design_plot.pdf}}}
\subfigure[Dataset B: Supergeo design]{\label{fig:B_supergeo_design}{\includegraphics[trim={0 2cm 2.5cm 2cm},clip,width=0.44\textwidth]{figs/new_pilot/design_plot.pdf}}}

\medskip
\raggedright
{\small\textit{Note}: Each dot represents one (super)geo and the dashed line between two dots implies that these two (super)geos are matched with ``left'' and ``right'' representing the two ``sides'' of each (super)geo pair. The $x$-axis is the scaled value of total pretest or test response. The blue circle represents a single geo, the orange cross represents a supergeo consisting of 2 geos, the green square represents a supergeo consisting of 3 geos.}
\caption{Matching quality achieved by different designs on datasets A and B.}
\label{fig:design}
\end{figure*}

% \begin{figure*}[!ht]
% \centering
% \subfigure[A: $\hat{\theta}$ of Pairs]{\label{fig:A_no_trim}{\includegraphics[width=0.24\textwidth]{figs/old_pilot_baseline/heavy_up_5e4.pdf}}}
% \subfigure[A: $\hat{\theta}^{\mathrm{trim}}$ of Pairs ]{\label{fig:A_with_trim}{\includegraphics[width=0.24\textwidth]{figs/old_pilot_baseline/heavy_up_5e4_trimmed.pdf}}}
% \subfigure[A: $\hat{\theta}$ of Supergeo]{\label{fig:A_supergeo}{\includegraphics[width=0.24\textwidth]{figs/old_pilot/heavy_up_5e4.pdf}}}
% \subfigure[A: $\hat{\theta}^{\mathrm{trim}}$ of Supergeo]{\label{fig:A_supergeo_trim}{\includegraphics[width=0.24\textwidth]{figs/old_pilot/heavy_up_5e4_trimmed.pdf}}}
% \subfigure[B: $\hat{\theta}$ of Pairs]{\label{fig:B_no_trim}{\includegraphics[width=0.24\textwidth]{figs/new_pilot_baseline/heavy_up_1e5.pdf}}}
% \subfigure[B: $\hat{\theta}^{\mathrm{trim}}$ of Pairs ]{\label{fig:B_with_trim}{\includegraphics[width=0.24\textwidth]{figs/new_pilot_baseline/heavy_up_1e5_trimmed.pdf}}}
% \subfigure[B: $\hat{\theta}$ of Supergeo]{\label{fig:B_supergeo}{\includegraphics[width=0.24\textwidth]{figs/new_pilot/heavy_up_1e5.pdf}}}
% \subfigure[B: $\hat{\theta}^{\mathrm{trim}}$ of Supergeo]{\label{fig:B_supergeo_trim}{\includegraphics[width=0.24\textwidth]{figs/new_pilot/heavy_up_1e5_trimmed.pdf}}}
% \caption{The histogram of the estimates $\hat{\theta}$ when the true $\theta=1$ is homogeneous. The red vertical line shows the true $\theta$.}
% \label{fig:homogeneous}
% \end{figure*}

\begin{figure*}[!ht]
\centering
\begin{tabular}{@{} c|c @{}}
\subfigure[A: $\hat{\theta}$ of Pairs ]{\label{fig:A_no_trim}{\includegraphics[width=0.24\textwidth]{figs/old_pilot_baseline/heavy_up_5e4.pdf}}}
\subfigure[A: $\hat{\theta}$ of Supergeo]{\label{fig:A_supergeo}{\includegraphics[width=0.24\textwidth]{figs/old_pilot/heavy_up_5e4.pdf}}}
&
\subfigure[A: $\hat{\theta}^{\mathrm{trim}}$ of Pairs ]{\label{fig:A_with_trim}{\includegraphics[width=0.24\textwidth]{figs/old_pilot_baseline/heavy_up_5e4_trimmed.pdf}}}
\subfigure[A: $\hat{\theta}^{\mathrm{trim}}$ of Supergeo]{\label{fig:A_supergeo_trim}{\includegraphics[width=0.24\textwidth]{figs/old_pilot/heavy_up_5e4_trimmed.pdf}}}
\\ \hline
\subfigure[B: $\hat{\theta}$ of Pairs]{\label{fig:B_no_trim}{\includegraphics[width=0.24\textwidth]{figs/new_pilot_baseline/heavy_up_1e5.pdf}}}
\subfigure[B: $\hat{\theta}$ of Supergeo]{\label{fig:B_supergeo}{\includegraphics[width=0.24\textwidth]{figs/new_pilot/heavy_up_1e5.pdf}}}
&
\subfigure[B: $\hat{\theta}^{\mathrm{trim}}$ of Pairs ]{\label{fig:B_with_trim}{\includegraphics[width=0.24\textwidth]{figs/new_pilot_baseline/heavy_up_1e5_trimmed.pdf}}}
\subfigure[B: $\hat{\theta}^{\mathrm{trim}}$ of Supergeo]{\label{fig:B_supergeo_trim}{\includegraphics[width=0.24\textwidth]{figs/new_pilot/heavy_up_1e5_trimmed.pdf}}}
\end{tabular}

\medskip
\raggedright
{\small\textit{Note}: The dataset is either A or B, the estimator is either the empirical estimator $\hat{\theta}$ or the trimmed match estimator $\hat{\theta}^{\mathrm{trim}}$, and the experimental design is either the matched pairs design (abbreviated as Pairs) or the supergeo design (abbreviated as Supergeo). The red vertical line corresponds to the true value of $\theta$.}
\caption{The histograms of the estimates under homogeneous iROAS.}
\label{fig:homogeneous}
\end{figure*}






\begin{figure*}[!ht]
\centering
\begin{tabular}{@{} c|c @{}}
\subfigure[A: $\hat{\theta}$ of Pairs]{\label{fig:A_hetero_no_trim}{\includegraphics[width=0.24\textwidth]{figs/old_pilot_baseline/hetero_prop_5e4.pdf}}}
\subfigure[A: $\hat{\theta}$ of Supergeo]{\label{fig:A_hetero_supergeo}{\includegraphics[width=0.24\textwidth]{figs/old_pilot/hetero_prop_5e4.pdf}}}
&
\subfigure[A: $\hat{\theta}^{\mathrm{trim}}$ of Pairs ]{\label{fig:A_hetero_with_trim}{\includegraphics[width=0.24\textwidth]{figs/old_pilot_baseline/hetero_prop_5e4_trimmed.pdf}}}
\subfigure[A: $\hat{\theta}^{\mathrm{trim}}$ of Supergeo]{\label{fig:A_hetero_supergeo_trim}{\includegraphics[width=0.24\textwidth]{figs/old_pilot/hetero_prop_5e4_trimmed.pdf}}}
\\ \hline
\subfigure[B: $\hat{\theta}$ of Pairs]{\label{fig:B_hetero_no_trim}{\includegraphics[width=0.24\textwidth]{figs/new_pilot_baseline/hetero_prop_1e5.pdf}}}
\subfigure[B: $\hat{\theta}$ of Supergeo]{\label{fig:B_hetero_supergeo}{\includegraphics[width=0.24\textwidth]{figs/new_pilot/hetero_prop_1e5.pdf}}}
&
\subfigure[B: $\hat{\theta}^{\mathrm{trim}}$ of Pairs ]{\label{fig:B_hetero_with_trim}{\includegraphics[width=0.24\textwidth]{figs/new_pilot_baseline/hetero_prop_1e5_trimmed.pdf}}}
\subfigure[B: $\hat{\theta}^{\mathrm{trim}}$ of Supergeo]{\label{fig:B_hetero_supergeo_trim}{\includegraphics[width=0.24\textwidth]{figs/new_pilot/hetero_prop_1e5_trimmed.pdf}}}
\end{tabular}

\medskip
\centering
{\small\textit{Note}:  The figures are shown in the same format as Figure~\ref{fig:homogeneous}.}
\caption{The histograms of the estimates under heterogeneous iROAS, where the heterogeneous iROAS has an additive term that is proportional to the geo size.}
\label{fig:heterogeneous}
\end{figure*}


\clearpage
\newpage


\section{Additional figures}\label{sec:other_figs}
See Figures~\ref{fig:design}, \ref{fig:homogeneous}, and~\ref{fig:heterogeneous} for additional empirical results referred to in the paper.



% \begin{figure*}[!h]
%   \centering
%   \begin{subfigure}[t]{.49\linewidth}
%     \centering\includegraphics[width=\linewidth]{figs/match_baseline_scaled.png}
%     \caption{Geo matching design}
%   \end{subfigure}
%   \begin{subfigure}[t]{.49\linewidth}
%     \centering\includegraphics[width=\linewidth]{figs/match_supergeo_scaled.png}
%     \caption{Supergeo matching design}
%   \end{subfigure}
%   \caption{Pairing results of different designs. Each dot represents one unit and the dashed line between two dots implies these two units are paired. The x-axis is the scaled volume of pretest responses based on some real customer study.}\label{fig:matching}
% \end{figure*}

% \begin{figure*}[!h]
%   \centering
%   \begin{subfigure}[t]{.3\linewidth}
%     \centering\includegraphics[width=\linewidth]{figs/baseline_no_trim.png}
%     \caption{Baseline (no trimming)}
%   \end{subfigure}
%   \begin{subfigure}[t]{.3\linewidth}
%     \centering\includegraphics[width=\linewidth]{figs/baseline_trim.png}
%     \caption{Baseline (with trimming)}
%   \end{subfigure}
%   \begin{subfigure}[t]{.3\linewidth}
%     \centering\includegraphics[width=\linewidth]{figs/supergeo.png}
%     \caption{Supergeo design}
%   \end{subfigure}
%   \caption{Homogeneous treatment effects}\label{fig:homogeneous}
% \end{figure*}
% %\vspace{-1mm}

% \begin{figure*}[!h]
%   \centering
%   \begin{subfigure}[t]{.3\linewidth}
%     \centering\includegraphics[width=\linewidth]{figs/baseline_no_trim_heterogeneous.png}
%     \caption{Baseline (no trimming)}
%   \end{subfigure}
%   \begin{subfigure}[t]{.3\linewidth}
%     \centering\includegraphics[width=\linewidth]{figs/baseline_trim_heterogeneous.png}
%     \caption{Baseline (with trimming)}
%   \end{subfigure}
%   \begin{subfigure}[t]{.3\linewidth}
%     \centering\includegraphics[width=\linewidth]{figs/supergeo_heterogeneous.png}
%     \caption{Supergeo design}
%   \end{subfigure}
%   \caption{Heterogeneous treatment effects (with the iROAS proportional to the uninfluenced response)}\label{fig:heterogeneous}
% \end{figure*}
\section{Additional evaluations}\label{sec:other_hetero}
% \subsection{Illustration of pairing quality}\label{sec:pairing}
% In Figure~\ref{fig:design} we show the pairing results of the matching pairs design and the supergeo design in Section~\ref{sec:eval} on the two datasets A and B. From the figures, we can see that on pretest data the supergeos are matched much better than the matched pairs design. On test data the supergeos are matched slightly worse since there are temporal noises between the pretest phase and the test phase. We also observe that the temporal noises are larger on dataset A comparing to dataset B, and we conjecture this is the reason why supergeo design performs worse on dataset A, as shown in Table~\ref{tab:homogeneous}.

% We remark that there are may 1-1 pairs in the supergeo design, and this is because we used the per-geo heuristic (see Section~\ref{sec:heuristics}) that only includes the well-matched supergeo pairs that use the largest geos. The smaller geos are already matched well by the 1-1 pairs. \Shunhua{@Aiyou, this is why there are many size 1 supergeo.}

% \begin{figure*}[!tb]
% \centering
% \subfigure[Dataset A: matched pairs design]{\label{fig:A_baseline_design}{\includegraphics[trim={0 0 2.5cm 0},clip,width=0.48\textwidth]{figs/old_pilot_baseline/design_plot.pdf}}}
% \subfigure[Dataset A: Supergeo design]{\label{fig:A_supergeo_design}{\includegraphics[trim={0 0 2.5cm 0},clip,width=0.48\textwidth]{figs/old_pilot/design_plot.pdf}}}
% \subfigure[Dataset B: matched pairs design]{\label{fig:B_baseline_design}{\includegraphics[trim={0 0 2.5cm 0},clip,width=0.48\textwidth]{figs/new_pilot_baseline/design_plot.pdf}}}
% \subfigure[Dataset B: Supergeo design]{\label{fig:B_supergeo_design}{\includegraphics[trim={0 0 2.5cm 0},clip,width=0.48\textwidth]{figs/new_pilot/design_plot.pdf}}}
% \caption{Pairing results of different designs on dataset A and B. Each dot represents one (super)geo and the dashed line between two dots implies these two (super)geos are paired. The x-axis is the scaled value of total pretest responses. The blue circle represents a single geo, the orange cross represents a supergeo with 2 geos, and the green square represents a supergeo with 3 geos.
% }
% \label{fig:design}
% \end{figure*}

% \subsection{Supergeo with trimming}
% We remark that we can combine the supergeo design proposed in this paper with the existing post-analysis tool trimmed match. The evaluation results of supergeo design with trimming are shown in Table~\ref{tab:homogeneous_extended} and \ref{tab:heterogeneous_extended}. We can see that the results of supergeo design with trimming is in the middle between matched pairs design with trimming and supergeo design without trimming.

% \begin{table}[!h]
% \centering
% \begin{tabular}{r@{\ }l|cc|cc}
%     \toprule
%     & & \multicolumn{2}{c|}{A} & \multicolumn{2}{c}{B} \\
%     \cmidrule(r){3-6}
%     & & RMSE & Cov. & RMSE & Cov. \\
%     \midrule
%     (a) & No trimming & $0.97$ & $74\%$ & $2.03$ & $77\%$ \\
%     (b) & Supergeo & $0.42$ & $79\%$ & $0.30$ & $78\%$ \\
%     (c) & With trimming & $0.33$ & $76\%$ & $0.42$ & $79\%$ \\
%     (d) & Supergeo with trimming & $0.34$ & $78\%$ & $0.37$ & $72\%$ \\
%     \bottomrule
%   \end{tabular}
% \caption{Extended version of Table~\ref{tab:homogeneous} (homogeneous) including supergeo design with trimming.}
% \label{tab:homogeneous_extended}
% \end{table}

% \begin{table}[!h]
% \centering
% \begin{tabular}{r@{\ }l|cc|cc}
%     \toprule
%     & & \multicolumn{2}{c|}{A} & \multicolumn{2}{c}{B} \\
%     \cmidrule(r){3-6}
%     & & RMSE & Cov. & RMSE & Cov. \\
%     \midrule
%     (a) & (Pair Match?) No trimming & $1.01$ & $75\%$ & $2.17$ & $77\%$ \\
%     (b) & Supergeo & $0.38$ & $81\%$ & $0.47$ & $72\%$ \\
%     (c) & (Pair Match?) With trimming & $0.58$ & $47\%$ & $0.65$ & $49\%$ \\
%     (d) & Supergeo with trimming & $0.54$ & $42\%$ & $0.52$ & $52\%$ \\
%     \bottomrule
%   \end{tabular}
% \caption{Extended version of Table~\ref{tab:heterogeneous} (heterogeneous) including supergeo design with trimming.}
% \label{tab:heterogeneous_extended}
% \end{table}

%\subsection{Other heterogeneous model}\label{sec:other_hetero}
In this section we report additional results based on an alternative model for heterogeneous $\theta_g$. We set $\theta_g = 1 + u_g$, where $u_g$ is an independent random value generated from the uniform distribution over $[-0.25, 0.25]$. The results are shown in Table~\ref{tab:heterogeneous_unif} and Figure~\ref{fig:heterogeneous_unif}.

% \begin{table}[!h]
% \centering
% \begin{tabular}{l|l|cc|cc}
%     \toprule
%     \multirow{2}{*}{Est.} & \multirow{2}{*}{Design} & \multicolumn{2}{c|}{Dataset A} & \multicolumn{2}{c}{Dataset B} \\
%     \cmidrule(r){3-6}
%     & & RMSE & Cov. & RMSE & Cov. \\
%     \midrule
%     \multirow{2}{*}{$\hat{\theta}$} & Pairs & $0.96$ & $74\%$ & $2.04$ & $77\%$ \\
%      & Supergeo & $0.41$ & $78\%$ & $0.29$ & $79\%$ \\
%     \midrule
%     \multirow{2}{*}{$\hat{\theta}^{\mathrm{trim}}$} & Pairs & $0.32$ & $76\%$ & $0.43$ & $78\%$ \\
%      & Supergeo & $0.32$ & $77\%$ & $0.41$ & $71\%$ \\
%     \bottomrule
%   \end{tabular}
% \caption{Evaluation data under heterogeneous iROAS, where $\theta_g$ is 1 plus a uniformly random noise. The table is shown in the same format as Table~\ref{tab:homogeneous}.}
% \label{tab:heterogeneous_unif}
% \end{table}
\begin{table}[!h]
\centering
\begin{tabular}{l|l|cc|cc}
    \toprule
    \multirow{2}{*}{Est.} & \multirow{2}{*}{Design} & \multicolumn{2}{c|}{Dataset A} & \multicolumn{2}{c}{Dataset B} \\
    \cmidrule(r){3-6}
    & & RMSE & Bias & RMSE & Bias \\
    \midrule
    \multirow{2}{*}{$\hat{\theta}$} & Pairs & $0.96$ & $0.032$ & $2.04$ & $0.119$ \\
     & Supergeo & $0.41$ & $0.001$ & $0.29$ & $0.006$ \\
    \midrule
    \multirow{2}{*}{$\hat{\theta}^{\mathrm{trim}}$} & Pairs & $0.32$ & $0.002$ & $0.43$ & $0.057$ \\
     & Supergeo & $0.32$ & $0.012$ & $0.41$ & $0.025$ \\
    \bottomrule
  \end{tabular}
  
\medskip
\centering
{\small\textit{Note}: The table is shown in the same format as Table~\ref{tab:homogeneous}.}
\caption{Empirical results under heterogeneous iROAS, where $\theta_g$ is 1 plus a uniformly random noise.}
\label{tab:heterogeneous_unif}
\end{table}

\begin{figure*}[!ht]
\centering
\begin{tabular}{@{} c|c @{}}
\subfigure[A: $\hat{\theta}$ of Pairs]{\label{fig:A_hetero_unif_no_trim}{\includegraphics[width=0.24\textwidth]{figs/old_pilot_baseline/hetero_unif_5e4.pdf}}}
\subfigure[A: $\hat{\theta}$ of Supergeo]{\label{fig:A_hetero_unif_supergeo}{\includegraphics[width=0.24\textwidth]{figs/old_pilot/hetero_unif_5e4.pdf}}}
&
\subfigure[A: $\hat{\theta}^{\mathrm{trim}}$ of Pairs ]{\label{fig:A_hetero_unif_with_trim}{\includegraphics[width=0.24\textwidth]{figs/old_pilot_baseline/hetero_unif_5e4_trimmed.pdf}}}
\subfigure[A: $\hat{\theta}^{\mathrm{trim}}$ of Supergeo]{\label{fig:A_hetero_unif_supergeo_trim}{\includegraphics[width=0.24\textwidth]{figs/old_pilot/hetero_unif_5e4_trimmed.pdf}}}
\\ \hline
\subfigure[B: $\hat{\theta}$ of Pairs]{\label{fig:B_hetero_unif_no_trim}{\includegraphics[width=0.24\textwidth]{figs/new_pilot_baseline/hetero_unif_1e5.pdf}}}
\subfigure[B: $\hat{\theta}$ of Supergeo]{\label{fig:B_hetero_unif_supergeo}{\includegraphics[width=0.24\textwidth]{figs/new_pilot/hetero_unif_1e5.pdf}}}
&
\subfigure[B: $\hat{\theta}^{\mathrm{trim}}$ of Pairs ]{\label{fig:B_hetero_unif_with_trim}{\includegraphics[width=0.24\textwidth]{figs/new_pilot_baseline/hetero_unif_1e5_trimmed.pdf}}}
\subfigure[B: $\hat{\theta}^{\mathrm{trim}}$ of Supergeo]{\label{fig:B_hetero_unif_supergeo_trim}{\includegraphics[width=0.24\textwidth]{figs/new_pilot/hetero_unif_1e5_trimmed.pdf}}}
\end{tabular}

\medskip
\centering
{\small\textit{Note}: The figures are shown in the same format as Figure~\ref{fig:homogeneous}.}
\caption{The histogram of the estimates under heterogeneous iROAS, where the heterogeneous iROAS is 1 plus a uniformly random noise.}
\label{fig:heterogeneous_unif}
\end{figure*}



\section{Choosing supergeo sizes}\label{sec:size}
In this section we present additional theoretical results that speak to the robustness properties of randomized designs. In the spirit of \citet{banerjee2020theory}, we want to capture a scenario in which an approach to designing an experiment is presented to a potentially adversarial audience. If the design is ``over-optimized''---for example, when there are two large supergeos and only two possible treatment assignments---the adversary can easily come up with a model that compromises the experimental results. For instance, they can suggest the presence of an unobserved confounder that differs between the two supergeos and that is correlated with the outcome variable. When the treatment assignment is realized as a result of many independent random draws---as is the case with many smaller supergeo pairs---the job of the adversary becomes substantially more complicated.

We also present additional empirical results that illustrate the benefits of using the maximum allowed supergeo pair size within the range of $\{3,4,5\}$.

%\subsection{Adversarial perturbation model}\label{sec:adversarial_model}
%We first provide some intuitions about adversarial perturbations. Suppose we were to divide all the geos into two supergeos to miminize the variance in Eq.~\eqref{eq:variance}, then the treatment group and the control group are completely fixed (up to switching between treatment and control). If some bad event takes place on geos within the same group, e.g., an earthquake or a heat wave that affects a large portion of the country, then this bad event could corrupt our data significantly. The effect of such bad events could be mitigated if there are more pairs, because the random treatment and control assignments within different pairs could cancel out with each other. We capture this intuition in the following adversarial perturbation model.

\paragraph{Model.}
We follow the notation from Section~\ref{sec:homo_model} and maintain the fixed budget and the homogeneous $\theta$ assumptions. We modify the linear outcome model in Assumption~\ref{ass:linear} to include an adversarial noise $\delta_g$:
\[
R_g = \theta \cdot S_g + Z_g + \delta_g.
\]
\begin{itemize}
    \item The adversary picks the values of per-geo noises $\delta_g$ after observing the allocation of geos to supergeo pairs $\{(G_{k,+}, G_{k,-})\}_{k=1}^K$, but \emph{before} the realization of a particular random treatment assignment.
    %\item However, the adversary cannot access the random assignments to treatment and control. 
    \item The noise $\delta_g$ is bounded by $|\delta_g| \leq \eta \cdot Z_g$ for some constant $\eta$.
\end{itemize}


\paragraph{Expectation.}
Following an argument similar to that of Section~\ref{sec:homo_model} and defining $\delta_G := \sum_{g \in G} \delta_g$ for any $G \subseteq \mathcal{G}$, we have
\begin{align*}
    \hat{\theta} = &~ \frac{\sum_{g \in \mathcal{T}} R_g - \sum_{g' \in \mathcal{C}} R_{g'}}{\sum_{g \in \mathcal{T}} S_g - \sum_{g' \in \mathcal{C}} S_{g'}} \notag \\
    %= &~ \frac{\sum_{i \in \mathcal{T}} (\theta \cdot S_g + Z_g + \epsilon_g) - \sum_{g' \in \mathcal{C}} (\theta \cdot S_{g'} + Z_{g'} + \epsilon_{g'})}{\sum_{i \in \mathcal{T}} S_g - \sum_{g' \in \mathcal{C}} S_{g'}} \\
    = &~ \theta + \frac{\sum_{g \in \mathcal{T}} (Z_g + \delta_g) - \sum_{g' \in \mathcal{C}} (Z_{g'} + \delta_{g'})}{\sum_{g \in \mathcal{T}} S_g - \sum_{g' \in \mathcal{C}} S_{g'}} \notag \\
    = &~ \theta + \frac{1}{B} \cdot \sum_{k} A_{k} \cdot (Z_{G_{k,+}} + \delta_{G_{k,+}} - Z_{G_{k,-}} - \delta_{G_{k,-}}).
\end{align*}
Since the random draws $A_{k}$'s are independent of the noises, $\delta_g$'s, the empirical estimator, $\hat{\theta}$, is still unbiased regardless of the values of $\delta_g$'s.


\paragraph{Variance.}
For any supergeo design $\{(G_{k,+}, G_{k,-})\}_{k=1}^K$, estimator $\hat{\theta}$ is a function of $\delta_g$'s and its variance is:
\begin{align*}
    var[\hat{\theta}(\{\delta_g\})] = &~ \frac{1}{B^2} \cdot \E\Big[ \Big( \sum_{k} A_{k} \cdot (Z_{G_{k,+}} + \delta_{G_{k,+}} - Z_{G_{k,-}} - \delta_{G_{k,-}}) \Big)^2 \Big] \\
    = &~ \frac{1}{B^2} \cdot \sum_{k} (Z_{G_{k,+}} + \delta_{G_{k,+}} - Z_{G_{k,-}} - \delta_{G_{k,-}})^2,
\end{align*}
where the second equality follows from that the fact that random draws $A_{k}$'s are independent of the noises $\delta_g$'s as well as independent from each other.

Let's compute the largest possible variance that can be achieved by the adversary picking the values of $\delta_g$'s. Consider any supergeo pair $(G_{k,+}, {G_{k,-}})$, and assume w.l.o.g.~that $Z_{G_{k,+}} \geq Z_{G_{k,-}}$. The largest value of the variance is realized when the adversary picks $\delta_{G_{k,+}} = \eta \cdot Z_{G_{k,+}}$ and $\delta_{G_{k,-}} = - \eta \cdot Z_{G_{k,-}}$:
\begin{align}\label{eq:variance_adversarial}
    \max_{\{\delta_g\}}\ var[\hat{\theta}(\{\delta_g\})] = &~ \max_{\{\delta_g\}}\ \frac{1}{B^2} \cdot \sum_{k} (Z_{G_{k,+}} + \delta_{G_{k,+}} - Z_{G_{k,-}} - \delta_{G_{k,-}})^2 \notag \\
    = &~ \frac{1}{B^2} \cdot \sum_{k} ( |Z_{G_{k,+}} - Z_{G_{k,-}}| + \eta \cdot Z_{G_{k,+}} + \eta \cdot Z_{G_{k,-}})^2 \notag \\ 
    = &~ \frac{1}{B^2} \cdot \sum_{k} \Big( (Z_{G_{k,+}} - Z_{G_{k,-}})^2 + 2 \eta \cdot |Z_{G_{k,+}}^2 - Z_{G_{k,-}}^2| + \eta^2 \cdot (Z_{G_{k,+}} + Z_{G_{k,-}})^2 \Big).
\end{align}
Importantly, the third term, $\eta^2 \cdot (Z_{G_{k,+}} + Z_{G_{k,-}})^2$, penalizes supergeo pairs that are too large. This motivates the decision to construct many smaller supergeo pairs.

\subsection{Empirical results}\label{sec:appendix-B-eval}
In this section we compare supergeo designs with different maximum allowed sizes of supergeo pairs. As discussed in Section~\ref{sec:supergeo_algo}, finding the optimal supergeo design can be computationally expensive especially when the subset size is large. To mitigate this issue, we work with a synthetic dataset consisting of 40 geos. We generate the synthetic data in the same way as in Section~5.1 of \cite{chen2021trimmed}.  %\Shunhua{Do we need to describe the synthetic model in details here?}\Aiyou{I think the reference to Sec-5.1 should be good.}

\paragraph{Pretest loss vs.~test loss.}
Figure~\ref{fig:sync_40_loss} shows the loss from Eq.~\eqref{eq:loss} as a function of the maximum allowed size of a supergeo pair which varies from 2 to 8. Recall that the loss is proportional to the variance of the estimated $\hat{\theta}$. When computing the loss, we approximate $Z_g$ either by the pretest response $R_g^{\mathrm{pre}}$ or the test response $R_g^{\mathrm{test}}$. Even though a larger subset size is effective at decreasing the pretest loss (i.e., the training loss), the test loss reaches the minimum when the subset size is 4 and remains roughly the same when we increase it further. This confirms our intuition that a large subset size leads to overfitting to the pretest data.%, and it is less effective to reduce the variance in the test phase.
\begin{figure}[!h]
\centering
\includegraphics[width=0.5\textwidth]{figs/synthetic_clr21_40/sync_40_loss.pdf}

\medskip
\raggedright
{\small\textit{Note}: Reporting the loss on the pretest data, $\sum_{k=1}^K (R_{G_{k,+}}^{\mathrm{pre}} - R_{G_{k,-}}^{\mathrm{pre}})^2$, and the loss on the test data $\sum_{k=1}^K (R_{G_{k,+}}^{\mathrm{test}} - R_{G_{k,-}}^{\mathrm{test}})^2$, as the subset size varies from 2 to 8. The loss is shown using a log scale.}
\caption{Loss on the pretest and test data.}
\label{fig:sync_40_loss}
\end{figure}

\paragraph{No adversarial noise.} We repeat the evaluations from Section~\ref{sec:evaluation_homogeneous} with $\theta_g = 1$ for all $g$ on the synthetic dataset with 40 geos. The results are shown in Table~\ref{tab:sync_40_geo}. We observe that the RMSE reaches the minimum when the subset size is 4 or 3 (depending on the estimator used) and does not decrease further when the subset size grows. 
\begin{table}[!h]
\centering
\begin{tabular}{l|c|cc|cc}
    \toprule
    & \multirow{2}{*}{\# pairs} & \multicolumn{2}{c|}{Empirical ($\hat{\theta}$)} & \multicolumn{2}{c}{Trimmed ($\hat{\theta}^{\mathrm{trim}}$)} \\
    \cmidrule(r){3-6}
    & & RMSE & Cov. & RMSE & Cov. \\
    \midrule
    Size 2 & 20 & $0.554$ & $76\%$ & $0.326$ & $70\%$ \\
    Size 3 & 14 & $0.163$ & $85\%$ & $\mathbf{0.059}$ & $76\%$ \\
    Size 4 & 10 & $\mathbf{0.100}$ & $76\%$ & $0.099$ & $62\%$ \\
    Size 5 & 8 & $0.104$ & $76\%$ & $0.105$ & $67\%$ \\
    Size 6 & 7 & $0.165$ & $75\%$ & $0.139$ & $63\%$ \\
    Size 7 & 6 & $0.105$ & $78\%$ & $0.114$ & $75\%$ \\
    Size 8 & 5 & $0.113$ & $69\%$ & $0.120$ & $63\%$ \\
    \bottomrule
  \end{tabular}
  
\medskip
\raggedright
{\small\textit{Note}: The coverage values (abbreviated as Cov.) are shown for nominally 80\% confidence intervals for supergeo designs with subset sizes from 2 to 8 on a synthetic dataset with 40 geos. We show the results of both the empirical estimator $\hat{\theta}$ and the trimmed match estimator $\hat{\theta}^{\mathrm{trim}}$. The confidence intervals are computed using the approximation by Student's $t$-distribution described in Section~5.2 of \citet{chen2022robust} (see Appendix~\ref{sec:inference}). We do not report the bias as it is close to zero (as is usually the case with homogeneous $\theta_g$'s).}
\caption{Root-mean-square errors (RMSE) of the estimates and the coverage.}
\label{tab:sync_40_geo}
\end{table}

\paragraph{With adversarial noise.}
We repeat the analysis with the adversarial noise added to the outcomes. Given a supergeo design $\{G_{k,+}, G_{k,-}\}_{k=1}^K$ and w.l.o.g.~assuming that $Z_{G_{k,+}} \geq Z_{G_{k,-}}$ for all $k$, we change $Z_{G_{k,+}}$ to $(1 + \eta) \cdot Z_{G_{k,+}}$ and change $Z_{G_{k,-}}$ to $(1 - \eta) \cdot Z_{G_{k,-}}$. We use $\eta = 0.05$ or $0.07$ in the evaluations.

The RMSEs of the empirical estimator are reported in Table~\ref{tab:sync_40_geo_adversarial}. We see that at first the RMSE declines as we increase the subset size which is due to the first two terms, $\sum_k (Z_{G_{k,+}} - Z_{G_{k,-}})^2 + 2 \eta \sum_k |Z_{G_{k,+}}^2 - Z_{G_{k,-}}^2|$, in the variance in Eq.~\eqref{eq:variance_adversarial}---the variance is lower when the pairs are better matched. However, as we further increase the subset size, the RMSE begins to grow due to the third term, $\eta^2 \cdot \sum_k (Z_{G_{k,+}} + Z_{G_{k,-}})^2$, which increases when the number of pairs decreases. In Figure~\ref{fig:sync_40_geo_adversarial} we include the histograms of the estimates $\hat{\theta}$ when $\eta = 0.07$.
% \begin{table}[!h]
% \centering
% \begin{tabular}{l|cc}
%     \toprule
%     & RMSE & Cov.  \\
%     \midrule
%     Size 2 & $0.90$ & $76\%$ \\
%     Size 3 & $0.55$ & $77\%$ \\
%     Size 4 & $0.52$ & $77\%$ \\
%     Size 5 & $0.55$ & $76\%$ \\
%     Size 6 & $0.63$ & $75\%$ \\
%     Size 7 & $0.61$ & $78\%$ \\
%     Size 8 & $0.65$ & $76\%$ \\
%     \bottomrule
%   \end{tabular}
% \caption{Root-mean-square errors of the estimates $\hat{\theta}$ and the coverage of nominally 80\% confidence intervals of supergeo designs with subset sizes from 2 to 8 under adversarial perturbations.}
% \label{tab:sync_40_geo_adversarial}
% \end{table}
% \begin{table}[!h]
% \centering
% \begin{tabular}{l|cc}
%     \toprule
%     & RMSE & Cov.  \\
%     \midrule
%     Size 2 & $0.79$ & $76\%$ \\
%     Size 3 & $0.44$ & $76\%$ \\
%     Size 4 & $0.39$ & $78\%$ \\
%     Size 5 & $0.42$ & $77\%$ \\
%     Size 6 & $0.50$ & $75\%$ \\
%     Size 7 & $0.47$ & $78\%$ \\
%     Size 8 & $0.50$ & $75\%$ \\
%     \bottomrule
%   \end{tabular}
% \caption{Root-mean-square errors of the estimates $\hat{\theta}$ and the coverage of nominally 80\% confidence intervals of supergeo designs with subset sizes from 2 to 8 under adversarial perturbations.}
% \label{tab:sync_40_geo_adversarial}
% \end{table}

\begin{table}[!h]
\centering
\begin{tabular}{l|c|c}
    \toprule
    & $\eta = 0.05$ & $\eta = 0.07$ \\
    \midrule
    Size 2 & $0.79$ & $0.90$ \\
    Size 3 & $0.44$ & $0.55$ \\
    Size 4 & $\mathbf{0.39}$ & $\mathbf{0.52}$ \\
    Size 5 & $0.42$ & $0.55$ \\
    Size 6 & $0.50$ & $0.63$ \\
    Size 7 & $0.47$ & $0.61$ \\
    Size 8 & $0.50$ &$0.65$ \\
    \bottomrule
  \end{tabular}
  
\medskip
\raggedright
{\small\textit{Note}: Reporting the RMSE of the empirical estimator, $\hat{\theta}$, in the adversarial perturbation model where $\eta = 0.05$ or $0.07$. We do not report the bias as it is close to zero (as is usually the case with homogeneous $\theta_g$'s).}
\caption{Root-mean-square errors (RMSE) of the empirical estimator.
%Root-mean-square errors (RMSE) of the estimates and the coverage (abbreviated as Cov.) of nominally 80\% confidence intervals of supergeo designs with subset sizes from 2 to 8 under adversarial perturbations. We use the empirical estimator $\hat{\theta}$.
}
\label{tab:sync_40_geo_adversarial}
\end{table}



% \begin{table}[!h]
% \centering
% \begin{tabular}{l|cc|cc}
%     \toprule
%     & \multicolumn{2}{c|}{$\eta = 0.05$} & \multicolumn{2}{c}{$\eta = 0.07$} \\
%     \cmidrule(r){2-5}
%     & RMSE & Cov. & RMSE & Cov. \\
%     \midrule
%     Size 2 & $0.79$ & $76\%$ & $0.90$ & $76\%$ \\
%     Size 3 & $0.44$ & $76\%$ & $0.55$ & $77\%$ \\
%     Size 4 & $0.39$ & $78\%$ & $0.52$ & $77\%$ \\
%     Size 5 & $0.42$ & $77\%$ & $0.55$ & $76\%$ \\
%     Size 6 & $0.50$ & $75\%$ & $0.63$ & $75\%$ \\
%     Size 7 & $0.47$ & $78\%$ & $0.61$ & $78\%$ \\
%     Size 8 & $0.50$ & $75\%$ &$0.65$ & $76\%$ \\
%     \bottomrule
%   \end{tabular}
% \caption{Root-mean-square errors (RMSE) of the estimates and the coverage (abbreviated as Cov.) of nominally 80\% confidence intervals of supergeo designs with subset sizes from 2 to 8 under adversarial perturbations. We use the empirical estimator $\hat{\theta}$.
% }
% \label{tab:sync_40_geo_adversarial}
% \end{table}

\begin{figure*}[!ht]
\centering
\subfigure[Size 2]{\label{fig:size_2_adv}{\includegraphics[width=0.23\textwidth]{figs/synthetic_clr21_40_large_num_pairs_20/heavy_up_1e7_adv_007.pdf}}}
\subfigure[Size 3]{\label{fig:size_3_adv}{\includegraphics[width=0.23\textwidth]{figs/synthetic_clr21_40_large_num_pairs_14/heavy_up_1e7_adv_007.pdf}}}
\subfigure[Size 4]{\label{fig:size_4_adv}{\includegraphics[width=0.23\textwidth]{figs/synthetic_clr21_40_large_num_pairs_10/heavy_up_1e7_adv_007.pdf}}}
\subfigure[Size 5]{\label{fig:size_5_adv}{\includegraphics[width=0.23\textwidth]{figs/synthetic_clr21_40_large_num_pairs_8/heavy_up_1e7_adv_007.pdf}}}
\subfigure[Size 6]{\label{fig:size_6_adv}{\includegraphics[width=0.23\textwidth]{figs/synthetic_clr21_40_large_num_pairs_7/heavy_up_1e7_adv_007.pdf}}}
\subfigure[Size 7]{\label{fig:size_7_adv}{\includegraphics[width=0.23\textwidth]{figs/synthetic_clr21_40_large_num_pairs_6/heavy_up_1e7_adv_007.pdf}}}
\subfigure[Size 8]{\label{fig:size_8_adv}{\includegraphics[width=0.23\textwidth]{figs/synthetic_clr21_40_large_num_pairs_5/heavy_up_1e7_adv_007.pdf}}}

\medskip
\centering
{\small\textit{Note}: The value of the perturbation parameter is $\eta = 0.07$.}
\caption{The histogram of the estimates $\hat{\theta}$ under adversarial perturbation.}
\label{fig:sync_40_geo_adversarial}
\end{figure*}
\section{Inference}\label{sec:inference}
This paper does not develop any novel approaches to statistical inference in matched pairs design. However, in this section we outline two ways that could be used to test hypotheses and construct confidence intervals.

\subsection{Approximation by Student's $t$-distribution}
We can use the approach from Section~5.2 of \citet{chen2022robust} which uses an approximation by Student's $t$-distribution to recover a confidence level for $\theta$.

%. For each supergeo pair $(G_{k,+}, G_{k,-})$ and a value of $\theta$ we can compute $\epsilon_k(\theta) := (R_{G_{k,+}} - \theta S_{G_{k,+}}) -  (R_{G_{k,-}} - \theta S_{G_{k,-}})$ if $G_{k,+}$ is in treatment and we flip the sign otherwise. We approximate the distribution of these errors $\{\epsilon_1(\theta),\dots, \epsilon_K(\theta)\}$ (after normalization) with a Student's $t$-distribution., and we compute the 80\% confidence intervals by computing the 80\% quantile of this distribution. 
%its error $\epsilon_k(\hat{\theta}) := (R_{G_{k,+}} - \hat{\theta} S_{G_{k,+}}) -  (R_{G_{k,-}} - \hat{\theta} S_{G_{k,-}})$ if $G_{k,+}$ is in treatment and we flip the sign otherwise. We approximate the distribution of the errors $\{\epsilon_1(\hat{\theta}),\dots, \epsilon_K(\hat{\theta})\}$ (after normalization) with a Student's $t$-distribution., and we compute the 80\% confidence intervals by computing the 80\% quantile of this distribution. 

%We then compute the coverage of the confidence interval which is defined as the percentage of the estimates that fall into the computed confidence intervals. In Table~\ref{tab:sync_40_geo} we observe that the estimation of the confidence interval can become less accurate when there are fewer supergeo pairs. This is more evident when using the empirical estimator, and less so when using the trimmed match estimator since many pairs are trimmed.
%We will compute the coverage of nominally 80\% confidence intervals, where the coverage is defined as the percentage of the estimates that fall into the computed confidence intervals. 
%Furthermore, the estimation of the confidence interval can become less accurate when there are fewer supergeo pairs. This justifies our choice of a small subset size in $\{3,4\}$.

\subsection{Permutation-based inference}
Alternatively, we can test the sharp null hypothesis of zero treatment effects across all geos. Under this null, the treatment assignment does not matter and the estimate $\hat{\theta}$ is randomly drawn from the distribution $\{\hat{\theta}_a\}_{a\in\mathcal{A}}$, where $\mathcal{A}$ is the set of all possible treatment assignments. This distribution can be approximated---under the null hypothesis---by repeatedly re-drawing different treatment assignments and computing the corresponding $\hat{\theta}_a$. The null hypothesis is then rejected if the original estimate, $\hat{\theta}$, falls beyond some quantiles (e.g.~10\% or 90\%) of the constructed distribution. 

These tests can also be inverted to produce a confidence interval, but that would require additional assumptions. For instance, for each candidate value $\theta^{*}$ and a null hypothesis $H_0\colon\theta=\theta^{*}$ we could first remove the effect of the treatment from the responses of the treated units (assuming that the null hypothesis holds and $\theta^{*}$ is indeed the true value of $\theta$). Then, we would re-draw the treatment assignment, $a\in\mathcal{A}$, inject the effects (still assuming $\theta=\theta^{*}$) into the responses of the newly treated units, and re-estimate $\hat{\theta}_a$. We would repeat this multiple times to construct the distribution and test the null hypothesis in the same way as we did for $\theta^{*}=0$. We would construct a confidence interval for $\theta$ as the set of all $\theta^{*}$ for which the corresponding $H_0$'s are not rejected at the desired level of statistical significance.
\section{MIP heuristics}\label{sec:heuristics}
%\Nick{I'm actually thinking that maybe we could push this section to the appendix in order to free up space for a couple of plots. WDYT?}\Aiyou{+1 good idea}
For larger $N$ (e.g.~210 US DMAs), it could take weeks
%\Aiyou{forever sounds informal.. is years or weeks more appropriate?}\Shunhua{I think weeks is fair. Bicheng has tried to run some experiments for a week, and it didn't finish.} \Aiyou{resolved then.}
to directly solve the covering MIP formulated in Section~\ref{sec:supergeo_algo}. To accelerate computation, we propose two heuristics that find an approximately optimal solution. Both heuristics reduce the dimensions of variable $x$ by reducing the size of the search space (candidate supergeo pairs).

\paragraph{Partition heuristic.}
The first heuristic that we call \emph{partition heuristic} randomly divides all geos into several partitions, and only includes subsets (supergeo pairs) $G$'s that use geos in the same partition. The number of partitions is a tunable parameter. This way we effectively reduce the number of variables. The advantage of this heuristic is that it is a randomized method, and particularly suitable for running multiple instances in parallel. The disadvantage is that some of the subsets---those that include geos from different partitions---are no longer considered.

\paragraph{Per-geo heuristic}
The second heuristic which we call \emph{per-geo heuristic} sorts the geos according to the magnitude of the pre-test response (which we use as an approximation of the uninfluenced response), and only considers the well-matched supergeo pairs that include the largest geos in the MIP---those geos are usually the hardest to match in the matched pairs design while the smaller geos do not necessarily benefit from supergeo design as much.

Specifically, for each of the $\beta$ geos with largest responses, among all subsets $G$ that contain this geo, we consider the $\alpha$ fraction of subsets that have the smallest score. Here $\beta$ and $\alpha$ are chosen to balance statistical performance with computational efficiency.
%\Aiyou{instead of simply saying r is a tunable parameter, what about adding some guidance, e.g. r =0.01 works well for our experiments? Also use a difference letter since r is used in other places?}\Shunhua{I changed it to $\alpha$. The choice of $\alpha$ really depends on the problem size. We usually do grid search from $1e-5$ to $1e-3$.}\Aiyou{I see. Maybe just say a small fraction, i.e. vaguely, so that it is computable?} 
The advantage of this heuristic is that it matches the largest geos well, and those are usually the geos that are responsible for the largest errors. This heuristic is also useful for increasing the resulting number of pairs in the design since it preserves the one-to-one pairs of the small geos that are already matched well (see, for example, the supergeo design shown in Figure~\ref{fig:design}). The disadvantage is that we do not consider supergeo pairs of sizes exceeding 2 consisting entirely of the smaller geos.
%This is the result of using the per-geo heuristic from Section~\ref{sec:heuristics} which aims to include the well-matched supergeo pairs containing some of the largest geos. The smaller geos are already matched well in one-to-one pairs.
 

In practice, these heuristics are not mutual exclusive. We usually run multiple designs---with different hyper-parameters and different random seeds---in parallel and pick the best design among them. Using these heuristics, we can approximately solve the MIP for $N = 210$ and maximum group size $u = 4$ within a few hours on a single machine. 
%\Shunhua{Does "within a few hours" sounds weak? If so, we could re-phrase this sentence.}\Aiyou{maybe say 'for large n, it may take forever to solve the MIP, and then say the heuristics helps reduce to a few hours?} \Bicheng{I think what Aiyou written is good. Should we further emphasize dozens of minutes are achievable on some standard single machine instead on super-computer. }\Shunhua{Done.} %dozens of minutes. 

%Figure~\ref{fig:matching} illustrates how well the pairs are matched in the geo matching design and the supergeo matching design.
\section{Proof of NP hardness}\label{sec:np_hard}
In this section we prove Theorem~\ref{thm:np_hard}.
\begin{theorem}[Supergeo design problem is NP-hard. Restatement of Theorem~\ref{thm:np_hard}]
The following problem is NP-hard: Given a set $\mathcal{G}$ of size $|\mathcal{G}|=3m$ and values $Z_g \in \mathbb{Z}^+$ for each $g \in \mathcal{G}$, for any subset $G \subseteq [3m]$ define
\[
\mathrm{score}(G) := \min_{\substack{G_+ \cup G_- = G \\ G_+ \cap G_- = \emptyset}} \Big( \sum_{i \in G_+} Z_{i} - \sum_{j \in G_-} Z_{j} \Big)^2.
\]
Determine whether $\mathcal{G}$ can be partitioned into $m$ disjoint sets $G_1, G_2, \dots, G_m$ such that for all $i \in [m]$, $|G_i|=3$ and $\mathrm{score}(G_i) = 0$ ($[m]$ stands for the set $\{1,2,\dots,m\}$).
%The following problem is NP hard: Given a set $\mathcal{G}$ of size $|\mathcal{G}|=3m$ and each $g \in \mathcal{G}$ has a value $Z_g \in \mathbb{Z}^+$. For any subset $G \subseteq [3m]$, define
%\[
%\mathrm{score}(G) := \min_{\substack{G_+ \cup G_- = G \\ G_+ \cap G_- = \emptyset}} \Big( (\sum_{i \in G_+} Z_{i}) - (\sum_{j \in G_-} Z_{j}) \Big)^2.
%\]
%The question is to determine whether $\mathcal{G}$ can be partitioned into $m$ disjoint sets $G_1, G_2, \cdots, G_m$ such that for all $i \in [m]$, $|G_i|=3$ and $\mathrm{score}(G_i) = 0$.
\end{theorem}
We prove the NP-hardness of our problem by a reduction from the following numerical 3-dimensional matching problem of \citet{gj90}.
\begin{theorem}[Numerical 3-dimensional matching problem is NP-hard, SP16 in Appendix A of \citet{gj90}]
The following problem is NP-hard: Given disjoint sets $W$, $X$, and $Y$, each containing $m$ elements, each element $a \in W \cup X \cup Y$ having size $s(a) \in \mathbb{Z}^+$, and given a bound $B \in \mathbb{Z}^+$. Determine whether $W \cup X \cup Y$ can be partitioned into $m$ disjoint sets $A_1, A_2, \dots, A_m$ such that for all $i \in [m]$, $A_i$ contains exactly one element from each of $W$, $X$, and $Y$, and $\sum_{a \in A_i} s(a) = B$.
\end{theorem}
\begin{proof}[Proof of Theorem~\ref{thm:np_hard}]
Given a numerical 3-dimensional matching instance, we transform it into a supergeo design instance as follows: Define $M := 1 + B + \sum_{a \in W \cup X \cup Y} s(a)$. Let $\mathcal{G} = W \cup X \cup Y$, and define the values as:
\begin{align*}
Z_w = &~ s(w) + M, \quad \forall w \in W, \\
Z_x = &~ s(x) + 3 M, \quad \forall x \in X, \\
Z_y = &~ B - s(y) + 4 M, \quad \forall y \in Y.
\end{align*}
We prove that $W \cup X \cup Y$ can be partitioned into $m$ disjoint sets with equal sums if and only if $\mathcal{G}$ can be partitioned into $m$ disjoint sets with zero scores.

On the one hand, if $W \cup X \cup Y$ can be partitioned into $m$ disjoint sets $A_1, A_2, \dots, A_m$ such that for all $i \in [m]$, $A_i = \{w_i, x_i, y_i\}$ where $w_i \in W$, $x_i \in X$, and $y_i \in Y$, and $s(w_i) + s(x_i) + s(y_i) = B$, then for all $i \in [m]$ let $G_i = A_i$, and we have
\begin{align*}
\mathrm{score}(G_i) = &~ (Z_{w_i} + Z_{x_i} - Z_{y_i})^2 \\
= &~ \Big((s(w_i) + M) + (s(x_i) + 3M) - (B - s(y_i) + 4M) \Big)^2 \\
= &~ \Big(s(w_i) + s(x_i) - B + s(y_i) \Big)^2 = 0.
\end{align*}

\noindent On the other hand, if $\mathcal{G}$ can be partitioned into $m$ disjoint sets $G_1, G_2, \dots, G_m$ such that for all $i \in [m]$, $|G_i|=3$ and $\mathrm{score}(G_i) = 0$, then we argue that each $G_i$ must be of the form $G_i = \{w_i, x_i, y_i\}$ where $w_i \in W$, $x_i \in X$, $y_i \in Y$, and $s(w_i) + s(x_i) + s(y_i) = B$. Denote $G_i = \{a_i, b_i, c_i\}$, and w.l.o.g.~assume that $\mathrm{score}(G_i) = (Z_{a_i} + Z_{b_i} - Z_{c_i})^2$. Note that the only way that three numbers $a, b, c \in \{1, 3, 4\}$ satisfy $a + b - c = 0$ is when $a = 1$, $b = 3$ (or $a = 3$, $b = 1$), and $c = 4$. This implies that unless $a_i \in W$, $b_i \in X$ (or $a_i \in X$, $b_i \in W$), and $c_i \in Y$, we must have that 
\[
Z_{a_i} + Z_{b_i} - Z_{c_i} = t \cdot M + s,
\]
where $t$ is some \emph{non-zero} integer and $|s| \leq B + s(a_i) + s(b_i) + s(c_i) < 1 + B + \sum_{a \in W \cup X \cup Y} s(a) = M$, and therefore $t \cdot M + s \neq 0$. Thus, $\mathrm{score}(G_i)$ can only possibly be zero when $a_i \in W$, $b_i \in X$ (or $a_i \in X$, $b_i \in W$), and $c_i \in Y$ in which case
\[
s(a_i) + s(b_i) + s(c_i) - B = Z_{a_i} + Z_{b_i} - Z_{c_i} = 0. \qedhere
\]
\end{proof}

\noindent This proof also shows that the supergeo design problem is strongly NP-hard since the numerical 3-dimentional matching problem is strongly NP-hard and in our proof the values $Z_g$'s of the constructed supergeo design instance are all polynomially bounded.

%%%%%%%%%%%%%%%%%%%%%%%%%%%%%%%%%%%%%%%%%%%%%%%%%%%%%%%%%%%%%%%%%%%%%%%%%%%%%%%
%%%%%%%%%%%%%%%%%%%%%%%%%%%%%%%%%%%%%%%%%%%%%%%%%%%%%%%%%%%%%%%%%%%%%%%%%%%%%%%


\end{document}


% This document was modified from the file originally made available by
% Pat Langley and Andrea Danyluk for ICML-2K. This version was created
% by Iain Murray in 2018, and modified by Alexandre Bouchard in
% 2019 and 2021 and by Csaba Szepesvari, Gang Niu and Sivan Sabato in 2022.
% Modified again in 2023 by Sivan Sabato and Jonathan Scarlett.
% Previous contributors include Dan Roy, Lise Getoor and Tobias
% Scheffer, which was slightly modified from the 2010 version by
% Thorsten Joachims & Johannes Fuernkranz, slightly modified from the
% 2009 version by Kiri Wagstaff and Sam Roweis's 2008 version, which is
% slightly modified from Prasad Tadepalli's 2007 version which is a
% lightly changed version of the previous year's version by Andrew
% Moore, which was in turn edited from those of Kristian Kersting and
% Codrina Lauth. Alex Smola contributed to the algorithmic style files.
