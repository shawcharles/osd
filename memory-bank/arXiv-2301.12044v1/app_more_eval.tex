\section{Additional evaluations}\label{sec:other_hetero}
% \subsection{Illustration of pairing quality}\label{sec:pairing}
% In Figure~\ref{fig:design} we show the pairing results of the matching pairs design and the supergeo design in Section~\ref{sec:eval} on the two datasets A and B. From the figures, we can see that on pretest data the supergeos are matched much better than the matched pairs design. On test data the supergeos are matched slightly worse since there are temporal noises between the pretest phase and the test phase. We also observe that the temporal noises are larger on dataset A comparing to dataset B, and we conjecture this is the reason why supergeo design performs worse on dataset A, as shown in Table~\ref{tab:homogeneous}.

% We remark that there are may 1-1 pairs in the supergeo design, and this is because we used the per-geo heuristic (see Section~\ref{sec:heuristics}) that only includes the well-matched supergeo pairs that use the largest geos. The smaller geos are already matched well by the 1-1 pairs. \Shunhua{@Aiyou, this is why there are many size 1 supergeo.}

% \begin{figure*}[!tb]
% \centering
% \subfigure[Dataset A: matched pairs design]{\label{fig:A_baseline_design}{\includegraphics[trim={0 0 2.5cm 0},clip,width=0.48\textwidth]{figs/old_pilot_baseline/design_plot.pdf}}}
% \subfigure[Dataset A: Supergeo design]{\label{fig:A_supergeo_design}{\includegraphics[trim={0 0 2.5cm 0},clip,width=0.48\textwidth]{figs/old_pilot/design_plot.pdf}}}
% \subfigure[Dataset B: matched pairs design]{\label{fig:B_baseline_design}{\includegraphics[trim={0 0 2.5cm 0},clip,width=0.48\textwidth]{figs/new_pilot_baseline/design_plot.pdf}}}
% \subfigure[Dataset B: Supergeo design]{\label{fig:B_supergeo_design}{\includegraphics[trim={0 0 2.5cm 0},clip,width=0.48\textwidth]{figs/new_pilot/design_plot.pdf}}}
% \caption{Pairing results of different designs on dataset A and B. Each dot represents one (super)geo and the dashed line between two dots implies these two (super)geos are paired. The x-axis is the scaled value of total pretest responses. The blue circle represents a single geo, the orange cross represents a supergeo with 2 geos, and the green square represents a supergeo with 3 geos.
% }
% \label{fig:design}
% \end{figure*}

% \subsection{Supergeo with trimming}
% We remark that we can combine the supergeo design proposed in this paper with the existing post-analysis tool trimmed match. The evaluation results of supergeo design with trimming are shown in Table~\ref{tab:homogeneous_extended} and \ref{tab:heterogeneous_extended}. We can see that the results of supergeo design with trimming is in the middle between matched pairs design with trimming and supergeo design without trimming.

% \begin{table}[!h]
% \centering
% \begin{tabular}{r@{\ }l|cc|cc}
%     \toprule
%     & & \multicolumn{2}{c|}{A} & \multicolumn{2}{c}{B} \\
%     \cmidrule(r){3-6}
%     & & RMSE & Cov. & RMSE & Cov. \\
%     \midrule
%     (a) & No trimming & $0.97$ & $74\%$ & $2.03$ & $77\%$ \\
%     (b) & Supergeo & $0.42$ & $79\%$ & $0.30$ & $78\%$ \\
%     (c) & With trimming & $0.33$ & $76\%$ & $0.42$ & $79\%$ \\
%     (d) & Supergeo with trimming & $0.34$ & $78\%$ & $0.37$ & $72\%$ \\
%     \bottomrule
%   \end{tabular}
% \caption{Extended version of Table~\ref{tab:homogeneous} (homogeneous) including supergeo design with trimming.}
% \label{tab:homogeneous_extended}
% \end{table}

% \begin{table}[!h]
% \centering
% \begin{tabular}{r@{\ }l|cc|cc}
%     \toprule
%     & & \multicolumn{2}{c|}{A} & \multicolumn{2}{c}{B} \\
%     \cmidrule(r){3-6}
%     & & RMSE & Cov. & RMSE & Cov. \\
%     \midrule
%     (a) & (Pair Match?) No trimming & $1.01$ & $75\%$ & $2.17$ & $77\%$ \\
%     (b) & Supergeo & $0.38$ & $81\%$ & $0.47$ & $72\%$ \\
%     (c) & (Pair Match?) With trimming & $0.58$ & $47\%$ & $0.65$ & $49\%$ \\
%     (d) & Supergeo with trimming & $0.54$ & $42\%$ & $0.52$ & $52\%$ \\
%     \bottomrule
%   \end{tabular}
% \caption{Extended version of Table~\ref{tab:heterogeneous} (heterogeneous) including supergeo design with trimming.}
% \label{tab:heterogeneous_extended}
% \end{table}

%\subsection{Other heterogeneous model}\label{sec:other_hetero}
In this section we report additional results based on an alternative model for heterogeneous $\theta_g$. We set $\theta_g = 1 + u_g$, where $u_g$ is an independent random value generated from the uniform distribution over $[-0.25, 0.25]$. The results are shown in Table~\ref{tab:heterogeneous_unif} and Figure~\ref{fig:heterogeneous_unif}.

% \begin{table}[!h]
% \centering
% \begin{tabular}{l|l|cc|cc}
%     \toprule
%     \multirow{2}{*}{Est.} & \multirow{2}{*}{Design} & \multicolumn{2}{c|}{Dataset A} & \multicolumn{2}{c}{Dataset B} \\
%     \cmidrule(r){3-6}
%     & & RMSE & Cov. & RMSE & Cov. \\
%     \midrule
%     \multirow{2}{*}{$\hat{\theta}$} & Pairs & $0.96$ & $74\%$ & $2.04$ & $77\%$ \\
%      & Supergeo & $0.41$ & $78\%$ & $0.29$ & $79\%$ \\
%     \midrule
%     \multirow{2}{*}{$\hat{\theta}^{\mathrm{trim}}$} & Pairs & $0.32$ & $76\%$ & $0.43$ & $78\%$ \\
%      & Supergeo & $0.32$ & $77\%$ & $0.41$ & $71\%$ \\
%     \bottomrule
%   \end{tabular}
% \caption{Evaluation data under heterogeneous iROAS, where $\theta_g$ is 1 plus a uniformly random noise. The table is shown in the same format as Table~\ref{tab:homogeneous}.}
% \label{tab:heterogeneous_unif}
% \end{table}
\begin{table}[!h]
\centering
\begin{tabular}{l|l|cc|cc}
    \toprule
    \multirow{2}{*}{Est.} & \multirow{2}{*}{Design} & \multicolumn{2}{c|}{Dataset A} & \multicolumn{2}{c}{Dataset B} \\
    \cmidrule(r){3-6}
    & & RMSE & Bias & RMSE & Bias \\
    \midrule
    \multirow{2}{*}{$\hat{\theta}$} & Pairs & $0.96$ & $0.032$ & $2.04$ & $0.119$ \\
     & Supergeo & $0.41$ & $0.001$ & $0.29$ & $0.006$ \\
    \midrule
    \multirow{2}{*}{$\hat{\theta}^{\mathrm{trim}}$} & Pairs & $0.32$ & $0.002$ & $0.43$ & $0.057$ \\
     & Supergeo & $0.32$ & $0.012$ & $0.41$ & $0.025$ \\
    \bottomrule
  \end{tabular}
  
\medskip
\centering
{\small\textit{Note}: The table is shown in the same format as Table~\ref{tab:homogeneous}.}
\caption{Empirical results under heterogeneous iROAS, where $\theta_g$ is 1 plus a uniformly random noise.}
\label{tab:heterogeneous_unif}
\end{table}

\begin{figure*}[!ht]
\centering
\begin{tabular}{@{} c|c @{}}
\subfigure[A: $\hat{\theta}$ of Pairs]{\label{fig:A_hetero_unif_no_trim}{\includegraphics[width=0.24\textwidth]{figs/old_pilot_baseline/hetero_unif_5e4.pdf}}}
\subfigure[A: $\hat{\theta}$ of Supergeo]{\label{fig:A_hetero_unif_supergeo}{\includegraphics[width=0.24\textwidth]{figs/old_pilot/hetero_unif_5e4.pdf}}}
&
\subfigure[A: $\hat{\theta}^{\mathrm{trim}}$ of Pairs ]{\label{fig:A_hetero_unif_with_trim}{\includegraphics[width=0.24\textwidth]{figs/old_pilot_baseline/hetero_unif_5e4_trimmed.pdf}}}
\subfigure[A: $\hat{\theta}^{\mathrm{trim}}$ of Supergeo]{\label{fig:A_hetero_unif_supergeo_trim}{\includegraphics[width=0.24\textwidth]{figs/old_pilot/hetero_unif_5e4_trimmed.pdf}}}
\\ \hline
\subfigure[B: $\hat{\theta}$ of Pairs]{\label{fig:B_hetero_unif_no_trim}{\includegraphics[width=0.24\textwidth]{figs/new_pilot_baseline/hetero_unif_1e5.pdf}}}
\subfigure[B: $\hat{\theta}$ of Supergeo]{\label{fig:B_hetero_unif_supergeo}{\includegraphics[width=0.24\textwidth]{figs/new_pilot/hetero_unif_1e5.pdf}}}
&
\subfigure[B: $\hat{\theta}^{\mathrm{trim}}$ of Pairs ]{\label{fig:B_hetero_unif_with_trim}{\includegraphics[width=0.24\textwidth]{figs/new_pilot_baseline/hetero_unif_1e5_trimmed.pdf}}}
\subfigure[B: $\hat{\theta}^{\mathrm{trim}}$ of Supergeo]{\label{fig:B_hetero_unif_supergeo_trim}{\includegraphics[width=0.24\textwidth]{figs/new_pilot/hetero_unif_1e5_trimmed.pdf}}}
\end{tabular}

\medskip
\centering
{\small\textit{Note}: The figures are shown in the same format as Figure~\ref{fig:homogeneous}.}
\caption{The histogram of the estimates under heterogeneous iROAS, where the heterogeneous iROAS is 1 plus a uniformly random noise.}
\label{fig:heterogeneous_unif}
\end{figure*}


