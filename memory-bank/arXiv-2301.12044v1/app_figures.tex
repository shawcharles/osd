\begin{figure*}[!ht]
\centering
\subfigure[Dataset A: Matched pairs design]{\label{fig:A_baseline_design}{\includegraphics[trim={0 2cm 2.5cm 2cm},clip,width=0.44\textwidth]{figs/old_pilot_baseline/design_plot.pdf}}}
\subfigure[Dataset A: Supergeo design]{\label{fig:A_supergeo_design}{\includegraphics[trim={0 2cm 2.5cm 2cm},clip,width=0.44\textwidth]{figs/old_pilot/design_plot.pdf}}}
\subfigure[Dataset B: Matched pairs design]{\label{fig:B_baseline_design}{\includegraphics[trim={0 2cm 2.5cm 2cm},clip,width=0.44\textwidth]{figs/new_pilot_baseline/design_plot.pdf}}}
\subfigure[Dataset B: Supergeo design]{\label{fig:B_supergeo_design}{\includegraphics[trim={0 2cm 2.5cm 2cm},clip,width=0.44\textwidth]{figs/new_pilot/design_plot.pdf}}}

\medskip
\raggedright
{\small\textit{Note}: Each dot represents one (super)geo and the dashed line between two dots implies that these two (super)geos are matched with ``left'' and ``right'' representing the two ``sides'' of each (super)geo pair. The $x$-axis is the scaled value of total pretest or test response. The blue circle represents a single geo, the orange cross represents a supergeo consisting of 2 geos, the green square represents a supergeo consisting of 3 geos.}
\caption{Matching quality achieved by different designs on datasets A and B.}
\label{fig:design}
\end{figure*}

% \begin{figure*}[!ht]
% \centering
% \subfigure[A: $\hat{\theta}$ of Pairs]{\label{fig:A_no_trim}{\includegraphics[width=0.24\textwidth]{figs/old_pilot_baseline/heavy_up_5e4.pdf}}}
% \subfigure[A: $\hat{\theta}^{\mathrm{trim}}$ of Pairs ]{\label{fig:A_with_trim}{\includegraphics[width=0.24\textwidth]{figs/old_pilot_baseline/heavy_up_5e4_trimmed.pdf}}}
% \subfigure[A: $\hat{\theta}$ of Supergeo]{\label{fig:A_supergeo}{\includegraphics[width=0.24\textwidth]{figs/old_pilot/heavy_up_5e4.pdf}}}
% \subfigure[A: $\hat{\theta}^{\mathrm{trim}}$ of Supergeo]{\label{fig:A_supergeo_trim}{\includegraphics[width=0.24\textwidth]{figs/old_pilot/heavy_up_5e4_trimmed.pdf}}}
% \subfigure[B: $\hat{\theta}$ of Pairs]{\label{fig:B_no_trim}{\includegraphics[width=0.24\textwidth]{figs/new_pilot_baseline/heavy_up_1e5.pdf}}}
% \subfigure[B: $\hat{\theta}^{\mathrm{trim}}$ of Pairs ]{\label{fig:B_with_trim}{\includegraphics[width=0.24\textwidth]{figs/new_pilot_baseline/heavy_up_1e5_trimmed.pdf}}}
% \subfigure[B: $\hat{\theta}$ of Supergeo]{\label{fig:B_supergeo}{\includegraphics[width=0.24\textwidth]{figs/new_pilot/heavy_up_1e5.pdf}}}
% \subfigure[B: $\hat{\theta}^{\mathrm{trim}}$ of Supergeo]{\label{fig:B_supergeo_trim}{\includegraphics[width=0.24\textwidth]{figs/new_pilot/heavy_up_1e5_trimmed.pdf}}}
% \caption{The histogram of the estimates $\hat{\theta}$ when the true $\theta=1$ is homogeneous. The red vertical line shows the true $\theta$.}
% \label{fig:homogeneous}
% \end{figure*}

\begin{figure*}[!ht]
\centering
\begin{tabular}{@{} c|c @{}}
\subfigure[A: $\hat{\theta}$ of Pairs ]{\label{fig:A_no_trim}{\includegraphics[width=0.24\textwidth]{figs/old_pilot_baseline/heavy_up_5e4.pdf}}}
\subfigure[A: $\hat{\theta}$ of Supergeo]{\label{fig:A_supergeo}{\includegraphics[width=0.24\textwidth]{figs/old_pilot/heavy_up_5e4.pdf}}}
&
\subfigure[A: $\hat{\theta}^{\mathrm{trim}}$ of Pairs ]{\label{fig:A_with_trim}{\includegraphics[width=0.24\textwidth]{figs/old_pilot_baseline/heavy_up_5e4_trimmed.pdf}}}
\subfigure[A: $\hat{\theta}^{\mathrm{trim}}$ of Supergeo]{\label{fig:A_supergeo_trim}{\includegraphics[width=0.24\textwidth]{figs/old_pilot/heavy_up_5e4_trimmed.pdf}}}
\\ \hline
\subfigure[B: $\hat{\theta}$ of Pairs]{\label{fig:B_no_trim}{\includegraphics[width=0.24\textwidth]{figs/new_pilot_baseline/heavy_up_1e5.pdf}}}
\subfigure[B: $\hat{\theta}$ of Supergeo]{\label{fig:B_supergeo}{\includegraphics[width=0.24\textwidth]{figs/new_pilot/heavy_up_1e5.pdf}}}
&
\subfigure[B: $\hat{\theta}^{\mathrm{trim}}$ of Pairs ]{\label{fig:B_with_trim}{\includegraphics[width=0.24\textwidth]{figs/new_pilot_baseline/heavy_up_1e5_trimmed.pdf}}}
\subfigure[B: $\hat{\theta}^{\mathrm{trim}}$ of Supergeo]{\label{fig:B_supergeo_trim}{\includegraphics[width=0.24\textwidth]{figs/new_pilot/heavy_up_1e5_trimmed.pdf}}}
\end{tabular}

\medskip
\raggedright
{\small\textit{Note}: The dataset is either A or B, the estimator is either the empirical estimator $\hat{\theta}$ or the trimmed match estimator $\hat{\theta}^{\mathrm{trim}}$, and the experimental design is either the matched pairs design (abbreviated as Pairs) or the supergeo design (abbreviated as Supergeo). The red vertical line corresponds to the true value of $\theta$.}
\caption{The histograms of the estimates under homogeneous iROAS.}
\label{fig:homogeneous}
\end{figure*}






\begin{figure*}[!ht]
\centering
\begin{tabular}{@{} c|c @{}}
\subfigure[A: $\hat{\theta}$ of Pairs]{\label{fig:A_hetero_no_trim}{\includegraphics[width=0.24\textwidth]{figs/old_pilot_baseline/hetero_prop_5e4.pdf}}}
\subfigure[A: $\hat{\theta}$ of Supergeo]{\label{fig:A_hetero_supergeo}{\includegraphics[width=0.24\textwidth]{figs/old_pilot/hetero_prop_5e4.pdf}}}
&
\subfigure[A: $\hat{\theta}^{\mathrm{trim}}$ of Pairs ]{\label{fig:A_hetero_with_trim}{\includegraphics[width=0.24\textwidth]{figs/old_pilot_baseline/hetero_prop_5e4_trimmed.pdf}}}
\subfigure[A: $\hat{\theta}^{\mathrm{trim}}$ of Supergeo]{\label{fig:A_hetero_supergeo_trim}{\includegraphics[width=0.24\textwidth]{figs/old_pilot/hetero_prop_5e4_trimmed.pdf}}}
\\ \hline
\subfigure[B: $\hat{\theta}$ of Pairs]{\label{fig:B_hetero_no_trim}{\includegraphics[width=0.24\textwidth]{figs/new_pilot_baseline/hetero_prop_1e5.pdf}}}
\subfigure[B: $\hat{\theta}$ of Supergeo]{\label{fig:B_hetero_supergeo}{\includegraphics[width=0.24\textwidth]{figs/new_pilot/hetero_prop_1e5.pdf}}}
&
\subfigure[B: $\hat{\theta}^{\mathrm{trim}}$ of Pairs ]{\label{fig:B_hetero_with_trim}{\includegraphics[width=0.24\textwidth]{figs/new_pilot_baseline/hetero_prop_1e5_trimmed.pdf}}}
\subfigure[B: $\hat{\theta}^{\mathrm{trim}}$ of Supergeo]{\label{fig:B_hetero_supergeo_trim}{\includegraphics[width=0.24\textwidth]{figs/new_pilot/hetero_prop_1e5_trimmed.pdf}}}
\end{tabular}

\medskip
\centering
{\small\textit{Note}:  The figures are shown in the same format as Figure~\ref{fig:homogeneous}.}
\caption{The histograms of the estimates under heterogeneous iROAS, where the heterogeneous iROAS has an additive term that is proportional to the geo size.}
\label{fig:heterogeneous}
\end{figure*}


\clearpage
\newpage


\section{Additional figures}\label{sec:other_figs}
See Figures~\ref{fig:design}, \ref{fig:homogeneous}, and~\ref{fig:heterogeneous} for additional empirical results referred to in the paper.



% \begin{figure*}[!h]
%   \centering
%   \begin{subfigure}[t]{.49\linewidth}
%     \centering\includegraphics[width=\linewidth]{figs/match_baseline_scaled.png}
%     \caption{Geo matching design}
%   \end{subfigure}
%   \begin{subfigure}[t]{.49\linewidth}
%     \centering\includegraphics[width=\linewidth]{figs/match_supergeo_scaled.png}
%     \caption{Supergeo matching design}
%   \end{subfigure}
%   \caption{Pairing results of different designs. Each dot represents one unit and the dashed line between two dots implies these two units are paired. The x-axis is the scaled volume of pretest responses based on some real customer study.}\label{fig:matching}
% \end{figure*}

% \begin{figure*}[!h]
%   \centering
%   \begin{subfigure}[t]{.3\linewidth}
%     \centering\includegraphics[width=\linewidth]{figs/baseline_no_trim.png}
%     \caption{Baseline (no trimming)}
%   \end{subfigure}
%   \begin{subfigure}[t]{.3\linewidth}
%     \centering\includegraphics[width=\linewidth]{figs/baseline_trim.png}
%     \caption{Baseline (with trimming)}
%   \end{subfigure}
%   \begin{subfigure}[t]{.3\linewidth}
%     \centering\includegraphics[width=\linewidth]{figs/supergeo.png}
%     \caption{Supergeo design}
%   \end{subfigure}
%   \caption{Homogeneous treatment effects}\label{fig:homogeneous}
% \end{figure*}
% %\vspace{-1mm}

% \begin{figure*}[!h]
%   \centering
%   \begin{subfigure}[t]{.3\linewidth}
%     \centering\includegraphics[width=\linewidth]{figs/baseline_no_trim_heterogeneous.png}
%     \caption{Baseline (no trimming)}
%   \end{subfigure}
%   \begin{subfigure}[t]{.3\linewidth}
%     \centering\includegraphics[width=\linewidth]{figs/baseline_trim_heterogeneous.png}
%     \caption{Baseline (with trimming)}
%   \end{subfigure}
%   \begin{subfigure}[t]{.3\linewidth}
%     \centering\includegraphics[width=\linewidth]{figs/supergeo_heterogeneous.png}
%     \caption{Supergeo design}
%   \end{subfigure}
%   \caption{Heterogeneous treatment effects (with the iROAS proportional to the uninfluenced response)}\label{fig:heterogeneous}
% \end{figure*}