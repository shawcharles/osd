\section{Empirical results}\label{sec:eval}
We evaluate the proposed design in the context of estimating the incremental return on ad spend (iROAS), where for each geographic region the ad spend and the response (e.g.~conversions or sales) are observed. For each geo $g \in \mathcal{G}$, we collect the spend $S_g[1],\dots,S_g[T]$ and responses $R_g[1],\dots,R_g[T]$ over $T$ time periods. We divide $T$ into a pretest phase $\{1,\dots,T_0\}$ and a test phase $\{T_0+1,\dots,T\}$, and refer to the aggregates $R_g^{\mathrm{pre}} := \sum_{t=1}^{T_0} R_g[t]$ and $R_g^{\mathrm{test}} := \sum_{t=T_0 + 1}^{T} R_g[t]$ as the (total) pretest and test responses respectively. We define the pretest spend $S_g^{\mathrm{pre}}$ and the test spend $S_g^{\mathrm{test}}$ similarly. We use the pretest response $R_g^{\mathrm{pre}}$ as the approximation of $Z_g$ which is unobserved and we use the algorithms described in Section~\ref{sec:supergeo_algo} to generate experimental designs. The tuning parameters $\ell$ (minimum size of a supergeo pair), $u$ (maximum size of a supergeo pair), and $\kappa$ (minimum number of supergeo pairs) are fixed at 2, 4, and 1 respectively.%\Aiyou{Shall we add that for computational reasons, the tuning parameters $u$ and $\kappa$ are fixed instead of cross validation}

We iterate over different treatment assignments performing what we call the ``half-synthetic'' evaluations---the underlying data comes from real online advertising settings with artificial treatment ``injected'' in the data. Specifically, at each iteration we increase (\emph{``heavy-up''}) the spend in each treated geo in the test period to $(1 + r) \cdot S_g^{\mathrm{test}}$ and leave the spend in the control geos unaffected.\footnote{The methods of this paper can also be applied to another commonly used type of experiment in which the spend in treated geos is set to zero (\emph{``go dark''}).} The response variable in a treated geo $g$ is increased to $R_g^{\mathrm{test}} + \theta_g \cdot r S_g^{\mathrm{test}}$, where $\theta_g$ is the iROAS for that geo that is chosen by us and depends on the particular evaluation as discussed further. The ratio $r$ is set in such a way that the total increase in the spend is equal to the fixed budget, $B$.

The underlying data comes from one of the two real datasets that we call ``A'' and ``B.'' In both datasets $\mathcal{G}$ corresponds to the set of 210 DMAs in the United States, and the unit of time is one day. We set both the pretest and test phases to be four weeks long. The budget, $B$, is chosen to roughly match the magnitude of the total spend in the absence of treatment effects. %\Shunhua{Check the total spends.}
The true response and spend values are scaled to be between $[0,1]$. For all experiments reported in this paper, the evaluation procedures are iterated $M=10000$ times across different treatment assignments. %To generate the supergeo design, we set the maximum supergeo pair size ($u$) to be 4 and minimum number of pairs ($\kappa$) to 80.\Aiyou{Shunhua or Nick: is kappa equal to 80?}
We run the heuristics from Appendix~\ref{sec:heuristics} with different parameters in parallel with a time limit of 3 hours, and then pick the design that leads to the smallest training loss. 

We compare the supergeo design to matched pairs design using the empirical estimator~(Eq.~\eqref{eq:theta_hat}) as well as the trimmed match estimator~(Eq.~\eqref{eq:theta_hat_trim}). The poorly matched pairs can be trimmed during either the design stage or the analysis stage (or both). This has the potential to greatly reduce the variance of the resulting estimator, but may introduce substantial bias if the trimmed geos differ from the remaining ones in terms of their response to treatment. In the evaluations we have conducted, the two approaches---trimming at the design stage or at the analysis stage---led to similar results with the latter being generally more effective since it can take into account not only the similarity of geos in the pretest period, but also their potential divergence in the test period due to noise. Consequently, we present the comparisons between the supergeo design which utilizes a simple empirical estimator and the matched pairs design which utilizes the trimmed match estimator from \citet{chen2022robust}. %which implements trimming at the analysis stage. 
However, the takeaways from the analysis would remain similar if the trimming was done at the design phase. We also emphasize that while the main advantage of the supergeo design is the ability to improve the matching quality without sacrificing any data, this design can be combined with trimming. This leads to a performance that is similar or better than that of the trimmed match estimator applied to matched pairs design, but may still lead to a bias when the effects are heterogeneous. 

\subsection{Homogeneous $\theta_g$}\label{sec:evaluation_homogeneous}
First, we present the comparisons between the different methodologies for the case of homogeneous $\theta_g=1$ for all $g\in\mathcal{G}$. In Table~\ref{tab:homogeneous} we report the \emph{root-mean-square error} (RMSE), $\sqrt{M^{-1}\sum_{m=1}^M (\hat{\theta}^{(m)} - \theta)^2}$, and the absolute bias, $|M^{-1}\sum_{m=1}^M \hat{\theta}^{(m)} - \theta|$, where $\hat{\theta}^{(m)}$ corresponds to the estimate computed in iteration $m$. %shows the RMSE and the bias of the matched pairs design and the supergeo design using both the empirical estimator ($\hat{\theta}$ as defined in Eq.~\eqref{eq:theta_hat}) and the trimmed match estimator ($\hat{\theta}^{\mathrm{trim}}$ as defined in Eq.~\eqref{eq:theta_hat_trim}). 
%$\hat{\theta}$ and the coverage of nominally 80\% confidence intervals, where the coverage is computed as the percentage of the estimates that fall into the computed confidence intervals.
See Figure~\ref{fig:homogeneous} in Appendix~\ref{sec:other_figs} for the histograms of the estimates.


% \begin{table}[!h]
% \centering
% \begin{tabular}{l|l|cc|cc}
%     \toprule
%     \multirow{2}{*}{Est.} & \multirow{2}{*}{Design} & \multicolumn{2}{c|}{Dataset A} & \multicolumn{2}{c}{Dataset B} \\
%     \cmidrule(r){3-6}
%     & & RMSE & Cov. & RMSE & Cov. \\
%     \midrule
%     \multirow{2}{*}{$\hat{\theta}$} & Pairs & $0.97$ & $74\%$ & $2.03$ & $77\%$ \\
%      & Supergeo & $0.42$ & $79\%$ & $0.30$ & $78\%$ \\
%     \midrule
%     \multirow{2}{*}{$\hat{\theta}^{\mathrm{trim}}$} & Pairs & $0.33$ & $76\%$ & $0.42$ & $79\%$ \\
%      & Supergeo & $0.34$ & $78\%$ & $0.37$ & $72\%$ \\
%     \bottomrule
%   \end{tabular}
% \caption{Evaluation data under homogeneous iROAS. Root-mean-square errors of the estimates $\hat{\theta}$ and the coverage (abbreviated as Cov.) of nominally 80\% confidence intervals. The estimator (abbreviated as Est.) is either the empirical estimator $\hat{\theta}$ or the trimmed match estimator $\hat{\theta}^{\mathrm{trim}}$, and the experimental design is either the matched pairs design (abbreviated as Pairs) or the supergeo design (abbreviated as Supergeo).}
% \label{tab:homogeneous}
% \end{table}
\begin{table}[!h]
\centering
\begin{tabular}{l|l|cc|cc}
    \toprule
    \multirow{2}{*}{Est.} & \multirow{2}{*}{Design} & \multicolumn{2}{c|}{Dataset A} & \multicolumn{2}{c}{Dataset B} \\
    \cmidrule(r){3-6}
    & & RMSE & Bias & RMSE & Bias \\
    \midrule
    \multirow{2}{*}{$\hat{\theta}$} & Pairs & $0.97$ & $0.022$ & $2.03$ & $0.107$ \\
     & Supergeo & $0.42$ & $0.005$ & $0.30$ & $0.006$ \\
    \midrule
    \multirow{2}{*}{$\hat{\theta}^{\mathrm{trim}}$} & Pairs & $0.33$ & $0.006$ & $0.42$ & $0.003$ \\
     & Supergeo & $0.34$ & $0.006$ & $0.37$ & $0.016$ \\
    \bottomrule
  \end{tabular}
  
\medskip
\raggedright
{\small\textit{Note}: The estimator (abbreviated as Est.) is either the empirical estimator $\hat{\theta}$ or the trimmed match estimator $\hat{\theta}^{\mathrm{trim}}$. The experimental design is either the matched pairs design (abbreviated as Pairs) %\Aiyou{We compare standard geo-based matched pairs and supergeo pairs, so use 'Geo' or  or 'Standard' (standard matched pairs) instead of 'Pairs'?}
or the supergeo design (abbreviated as Supergeo).}
\caption{Empirical results under homogeneous iROAS: the RMSE and the bias of the estimates.}
\label{tab:homogeneous}
\end{table}


As evident from the table, supergeo design almost uniformly outperforms the matched pairs design in terms of RMSE when both designs use the same estimator.\footnote{The only exception being the trimmed match estimator on dataset A, where the supergeo design is very slightly worse than the matched pairs design.} Moreover, the performance of the empirical estimator applied to supergeo design is comparable to that of the trimmed match estimator applied to the matched pairs design (supergeo design is better on dataset B, and worse on dataset A)---supergeo design is able to achieve similar performance without throwing out any data.
%\Cliff{Shouldn't we say that without trimming, our RMSE is way better than paired.  Trimming doesn't seem to help us much and even sometimes it hurts.  So we are able to get similar performance to trimming with pairs, but we are not throwing out any of the data (and we gave the negatives of throwing out data earlier).}

In Figure~\ref{fig:design_small} we report the matching quality achieved by the two designs in the pretest period. %\footnote{Note that there are still many one-to-one pairs in the supergeo design. This is the result of using the per-geo heuristic from Section~\ref{sec:heuristics} which aims to include the well-matched supergeo pairs containing some of the largest geos. The smaller geos are already matched well in one-to-one pairs.}\Aiyou{The footnote looks no longer needed if we keep $\kappa$ to permit singleton supergeos?}
We can see that supergeo design achieves a much better matching quality than the matched pairs design on the pretest data. Figure~\ref{fig:design} in Appendix~\ref{sec:other_figs} shows that on the test data, the matching quality of the supergeo design declines relative to the pretest period due to temporal noise. We also observe that this noise is more substantial in dataset A compared to dataset B, and we conjecture that this is the reason behind supergeo design performing worse on dataset A.

\begin{figure}[!tb]
\centering
\subfigure[Matched pairs design]{\label{fig:A_baseline_design_small}{\includegraphics[trim={0 18cm 2.5cm 3.9cm},clip,width=0.48\textwidth]{figs/old_pilot_baseline/design_plot.pdf}}}
\subfigure[Supergeo design]{\label{fig:A_supergeo_design_small}{\includegraphics[trim={0 18cm 2.5cm 3.9cm},clip,width=0.48\textwidth]{figs/old_pilot/design_plot.pdf}}}
%\vspace{-0.4cm}

\medskip
\raggedright
{\small\textit{Note}: Each dot represents one (super)geo and the dashed line between two dots implies that these two (super)geos are matched with ``left'' and ``right'' representing the two ``sides'' of each (super)geo pair. The $x$-axis is the scaled value of total pretest response. The blue circle represents a single geo, the orange cross represents a supergeo consisting of 2 geos, the green square represents a supergeo consisting of 3 geos.}
\caption{Matching quality achieved by different designs on dataset A in the pretest phase.
}
\label{fig:design_small}
%\vspace{-0.2cm}
\end{figure}



\subsection{Heterogeneous $\theta_g$}
Second, we consider the case when $\theta_g$'s can vary across geos. We modify $\theta_g$ by adding a compotent that is proportional to the uninfluenced response of a geo: $\theta_g = 1 + c \cdot Z_g/(N^{-1}\sum_{g} Z_g)$, where $c=0.2$ when applied to dataset A and $0.07$ when applied to dataset B. Following Eq.~\eqref{eq:theta_hetero_simplify}, the quantity of interest is computed as $\theta = \sum_g\theta_g \cdot S_g^{\mathrm{test}}/\sum_g S_g^{\mathrm{test}}$. %\Nick{@Shunhua why $\leq$?}\Shunhua{Because for dataset A we choose $c=0.2$ and for dataset B we choose a smaller $c=0.07$ (since dataset B is a bit more noisy). We could also state these two numbers separately.} is a small constant.

This model captures a scenario in which advertising is more effective in densely populated areas that often generate larger revenues. This assumption may or may not be adequate in a particular setting. What is important, is the fact that trimming of the poorly matched pairs will likely be non-random with respect to the distribution of $\theta_g$'s across geos. In Appendix~\ref{sec:other_hetero} we report the results from another simulation study with heterogeneous $\theta_g$'s---each $\theta_g$ equals $1$ plus a uniformly random noise. Under this model, the results are similar to those from the homogeneous model---while the sample size is reduced, the geos are trimmed in a way that is random with respect to the distribution of $\theta_g$'s, resulting in no substantial bias. This confirms our intuition from Section~\ref{sec:hetero_model}. %\Shunhua{I think the message in this footnote is important and we should put it in the main text.} %\Shunhua{Please check this sentence. We need a good explanation of why this is a reasonable model.}

Table~\ref{tab:heterogeneous} reports the RMSE and the absolute bias while Figure~\ref{fig:heterogeneous} in Appendix~\ref{sec:other_figs} shows the histograms of the estimates. Supergeo design---when combined with the empirical estimator---reduces the variance without introducing a bias and leads to the most precise estimates across the board. The trimmed match estimator is effective at reducing the RMSE of the matched pairs design by reducing the variance, but that is achieved at the cost of introducing a bias which may be substantial. Moreover, it does not improve the results over the empirical estimator when the supergeo design is used---the potential reduction in variance is not worth the price paid in the increased bias.%The trimmed match estimator may already performs pretty good using the empirical estimator, trimmed match estimator is no longer helpful. Our supergeo design with the simple empirical estimator achieves better RMSE than the trimmed match estimator, and this is consistent with our theoretical model in Section~\ref{sec:hetero_model}.

% \begin{table}[!h]
% \centering
% \begin{tabular}{l|l|cc|cc}
%     \toprule
%     \multirow{2}{*}{Est.} & \multirow{2}{*}{Design} & \multicolumn{2}{c|}{Dataset A} & \multicolumn{2}{c}{Dataset B} \\
%     \cmidrule(r){3-6}
%     & & RMSE & Cov. & RMSE & Cov. \\
%     \midrule
%     \multirow{2}{*}{$\hat{\theta}$} & Pairs & $1.01$ & $75\%$ & $2.17$ & $77\%$ \\
%      & Supergeo & $0.38$ & $81\%$ & $0.47$ & $72\%$ \\
%     \midrule
%     \multirow{2}{*}{$\hat{\theta}^{\mathrm{trim}}$} & Pairs & $0.58$ & $47\%$ & $0.65$ & $49\%$ \\
%      & Supergeo & $0.54$ & $42\%$ & $0.52$ & $52\%$ \\
%     \bottomrule
%   \end{tabular}
% \caption{Evaluation data under heterogeneous iROAS, where the heterogeneous iROAS is proportional to the geo size. The table is shown in the same format as Table~\ref{tab:homogeneous}.}
% \label{tab:heterogeneous}
% \end{table}

\begin{table}[!h]
\centering
\begin{tabular}{l|l|cc|cc}
    \toprule
    \multirow{2}{*}{Est.} & \multirow{2}{*}{Design} & \multicolumn{2}{c|}{Dataset A} & \multicolumn{2}{c}{Dataset B} \\
    \cmidrule(r){3-6}
    & & RMSE & Bias & RMSE & Bias \\
    \midrule
    \multirow{2}{*}{$\hat{\theta}$} & Pairs & $1.01$ & $0.040$ & $2.17$ & $0.140$ \\
     & Supergeo & $0.38$ & $0.003$ & $0.47$ & $0.004$ \\
    \midrule
    \multirow{2}{*}{$\hat{\theta}^{\mathrm{trim}}$} & Pairs & $0.58$ & $0.429$ & $0.65$ & $0.507$ \\
     & Supergeo & $0.54$ & $0.442$ & $0.52$ & $0.420$ \\
    \bottomrule
  \end{tabular}
  
\medskip
\centering
{\small\textit{Note}: The table is shown using the same format as is used in Table~\ref{tab:homogeneous}.}
\caption{Empirical results under heterogeneous iROAS.}
\label{tab:heterogeneous}
\end{table}


