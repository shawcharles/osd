\section{MIP heuristics}\label{sec:heuristics}
%\Nick{I'm actually thinking that maybe we could push this section to the appendix in order to free up space for a couple of plots. WDYT?}\Aiyou{+1 good idea}
For larger $N$ (e.g.~210 US DMAs), it could take weeks
%\Aiyou{forever sounds informal.. is years or weeks more appropriate?}\Shunhua{I think weeks is fair. Bicheng has tried to run some experiments for a week, and it didn't finish.} \Aiyou{resolved then.}
to directly solve the covering MIP formulated in Section~\ref{sec:supergeo_algo}. To accelerate computation, we propose two heuristics that find an approximately optimal solution. Both heuristics reduce the dimensions of variable $x$ by reducing the size of the search space (candidate supergeo pairs).

\paragraph{Partition heuristic.}
The first heuristic that we call \emph{partition heuristic} randomly divides all geos into several partitions, and only includes subsets (supergeo pairs) $G$'s that use geos in the same partition. The number of partitions is a tunable parameter. This way we effectively reduce the number of variables. The advantage of this heuristic is that it is a randomized method, and particularly suitable for running multiple instances in parallel. The disadvantage is that some of the subsets---those that include geos from different partitions---are no longer considered.

\paragraph{Per-geo heuristic}
The second heuristic which we call \emph{per-geo heuristic} sorts the geos according to the magnitude of the pre-test response (which we use as an approximation of the uninfluenced response), and only considers the well-matched supergeo pairs that include the largest geos in the MIP---those geos are usually the hardest to match in the matched pairs design while the smaller geos do not necessarily benefit from supergeo design as much.

Specifically, for each of the $\beta$ geos with largest responses, among all subsets $G$ that contain this geo, we consider the $\alpha$ fraction of subsets that have the smallest score. Here $\beta$ and $\alpha$ are chosen to balance statistical performance with computational efficiency.
%\Aiyou{instead of simply saying r is a tunable parameter, what about adding some guidance, e.g. r =0.01 works well for our experiments? Also use a difference letter since r is used in other places?}\Shunhua{I changed it to $\alpha$. The choice of $\alpha$ really depends on the problem size. We usually do grid search from $1e-5$ to $1e-3$.}\Aiyou{I see. Maybe just say a small fraction, i.e. vaguely, so that it is computable?} 
The advantage of this heuristic is that it matches the largest geos well, and those are usually the geos that are responsible for the largest errors. This heuristic is also useful for increasing the resulting number of pairs in the design since it preserves the one-to-one pairs of the small geos that are already matched well (see, for example, the supergeo design shown in Figure~\ref{fig:design}). The disadvantage is that we do not consider supergeo pairs of sizes exceeding 2 consisting entirely of the smaller geos.
%This is the result of using the per-geo heuristic from Section~\ref{sec:heuristics} which aims to include the well-matched supergeo pairs containing some of the largest geos. The smaller geos are already matched well in one-to-one pairs.
 

In practice, these heuristics are not mutual exclusive. We usually run multiple designs---with different hyper-parameters and different random seeds---in parallel and pick the best design among them. Using these heuristics, we can approximately solve the MIP for $N = 210$ and maximum group size $u = 4$ within a few hours on a single machine. 
%\Shunhua{Does "within a few hours" sounds weak? If so, we could re-phrase this sentence.}\Aiyou{maybe say 'for large n, it may take forever to solve the MIP, and then say the heuristics helps reduce to a few hours?} \Bicheng{I think what Aiyou written is good. Should we further emphasize dozens of minutes are achievable on some standard single machine instead on super-computer. }\Shunhua{Done.} %dozens of minutes. 

%Figure~\ref{fig:matching} illustrates how well the pairs are matched in the geo matching design and the supergeo matching design.